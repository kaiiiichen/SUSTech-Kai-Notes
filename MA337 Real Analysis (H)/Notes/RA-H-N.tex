\documentclass[12pt]{article}

\usepackage{amsmath, amssymb, amsthm}
\usepackage{scalerel,stackengine}
\usepackage{esint}
\usepackage{geometry}
\geometry{a4paper, margin=1in}
\usepackage{setspace}
\onehalfspacing
\usepackage{hyperref}
\hypersetup{colorlinks=true, linkcolor=blue, citecolor=magenta, urlcolor=cyan}
\usepackage{titlesec}
\usepackage[autostyle=true]{csquotes}
\usepackage{graphicx}

\titleformat{\section}
{\normalfont\bfseries\LARGE}
{\thesection}{1em}{}
\titleformat{\subsection}
{\normalfont\bfseries\Large}
{\thesubsection}{1em}{}
\titleformat{\subsubsection}
{\normalfont\itshape\bfseries\large}
{\thesubsubsection}{1em}{}

\theoremstyle{plain}
\newtheorem{theorem}{Theorem}[section]
\newtheorem{lemma}[theorem]{Lemma}
\newtheorem{proposition}[theorem]{Proposition}
\newtheorem{corollary}[theorem]{Corollary}
\newtheorem{definition}[theorem]{Definition}
\newtheorem{example}[theorem]{Example}
\newtheorem{remark}[theorem]{Remark}

\newcommand{\R}{\mathbb{R}}
\newcommand{\N}{\mathbb{N}}
\newcommand{\C}{\mathbb{C}}
\newcommand{\Z}{\mathbb{Z}}
\newcommand{\Q}{\mathbb{Q}}
\newcommand{\eps}{\varepsilon}

\numberwithin{equation}{section}

\usepackage{fancyhdr}

\pagestyle{fancy}
\fancyhf{}
\fancyhead[C]{MA337 Real Analysis (H) Notes}
\fancyfoot[R]{\hyperref[mytoc]{\large $\hookleftarrow$}}
\fancyfoot[C]{\thepage}
\setlength{\headheight}{14pt}
\setlength{\headsep}{20pt}
\setlength{\footskip}{25pt}
\fancypagestyle{plain}{
  \fancyhf{}
  \fancyhead[C]{MA337 Real Analysis (H) Notes}
  \fancyfoot[C]{\thepage}
}

\begin{document}

\title{Real Analysis Notes}
\author{Kai Chen\footnote{\href{https://math.sustech.edu.cn}{Department of Mathematis}, \href{https://www.sustech.edu.cn}{Southern University of Science and Technology (SUSTech)}}}
\date{\today}

\maketitle


These notes are compatible with the MA337 course (Fall 2025).

Main reference: \textit{Elements of the Theory of Functions and Functional Analysis} by A.N.Kolmogorov and S.V.Fomin.


\clearpage

\pagenumbering{roman}

\phantomsection

\label{mytoc}

\tableofcontents

\clearpage
\pagenumbering{arabic}

\section{Set Theory}

This section is a crash course on set theory.

\clearpage





\section{Continuous Maps}

\clearpage






\section{Metric Spaces}

\clearpage





\section{Compactness of Sets}

\clearpage





\section{General Measure Theory}
In this chapter, we embark on a journey into the fundamental concepts of measure.

This theory, developed by Henri Lebesgue at the beginning of the 20th century, provides a more robust and general framework for integration than the Riemann integral, allowing us to integrate a wider class of functions and providing powerful convergence theorems essential for modern analysis and probability theory. His theory was published originally in his dissertation \textit{Intégrale, longueur, aire ("Integral, length, area")} at the University of Nancy during 1902.

For a biography of Henri Lebesgue, please see Appendix A.1.

\clearpage

\subsection{Semi-ring, Ring, Algebra, $\sigma$-Algebra, Borel $\sigma$-Algebra}

\begin{definition}
  (Semi-Ring of Sets)

  A system of sets $S$ is called a \textbf{semi-ring} if it satisfies the following two axioms:
  \begin{enumerate}
    \item If $A, B \in S$, then $A \cap B \in S$.
    \item If $A, B \in S$, then there exist disjoint sets $A_1, A_2, \ldots, A_n \in S$ such that 
    
    $A \setminus B = \bigsqcup_{i=1}^n A_i$.
  \end{enumerate}
\end{definition}

\begin{example}
  (Semi-open cells in $\R^n$)

  $I_1, I_2, \ldots, I_n$: intervals in $\R$.
  $C:=I_1 \times I_2 \times \ldots \times I_n$ is called a \textbf{cell} in $\R^n$.

  \textbf{semi-open interval}: an interval that is closed at one end and open at the other end, e.g., $[a, b)$ or $(a, b]$.

  Let $S$ be the collection of all semi-open cells in $\R^d$ (not required to be finite!), i.e. $S = \{[a_1,b_1)\times\cdots\times[a_n,b_n): a_i,b_i\in\mathbb{R}, a_i<b_i\}$. Then $S$ is a semi-ring.

  \underline{Warning}: Be cautious about the directions of semi-open cells! The directions of all cells must coincide.
\end{example}

\begin{remark}
  \underline{Question}: Can we take all closed/open cells in $R^n$?

  \underline{Answer}: NO! For example, $[0,1] \cap [1,2] = \{1\}$, $(0,1) \setminus (1/2,1) = (0,1/2]$, both result in some elements not in the original system.

\end{remark}

\begin{proposition}
  If $S$ is a semi-ring, then
  \begin{enumerate}
    \item $\emptyset \in S$.
    \item Axiom 2 can be strengthened to: $\forall A \in S$, $\forall A_1, A_2, \ldots, A_n \in S, A_j \in A, \forall j, disjoint$, there exist disjoint sets $A_{m+1}, A_{m+2}, \ldots, A_s \in S$ such that $A = \bigsqcup_{i=1}^s A_i$.
  \end{enumerate}
  \begin{proof}
    1. $\emptyset = A \setminus A. \forall A \in S$.

    2. One can prove by induction on $m$: splitting the whole area $A$ into disjoint parts. It is easier to prove for the semi-ring $\{$all cells in $\R^n\}$.
  \end{proof}
\end{proposition}

\begin{remark}
  We now show that with axiom 1 and the strengthened condition above we could say $S$ is a semi-ring.
  \begin{proof}
    Now axiom 1 is satisfied.

    Suppose $A, B \in S$, then $A \setminus B = A \ (A \cap B)$. Let $A_1 = B, n = 1$. By our strengthened condition, one could find disjoint sets $A_2, A_3, \ldots, A_s \in S$, s.t. $A = \bigsqcup_{i=1}^s A_i$, i.e. $A \setminus B = \bigsqcup_{i=2}^s A_i$. $\checkmark$
  \end{proof}
  Thus, we have the following equivalent definition for semi-rings.
\end{remark}

\begin{definition}
  (Semi-Ring of Sets - Alternative Definition)

  A system of sets $S$ is called a \textbf{semi-ring} if it satisfies the following two axioms:
  \begin{enumerate}
    \item If $A, B \in S$, then $A \cap B \in S$.
    \item $\forall A \in S$, $\forall A_1, A_2, \ldots, A_n \in S, A_j \subset A, \forall j, disjoint$, there exist disjoint sets

      $A_{m+1}, A_{m+2}, \ldots, A_s \in S$ such that $A = \bigsqcup_{i=1}^s A_i$.
  \end{enumerate}
\end{definition}

\begin{definition}
  (Semi-ring with Unity)

  A semi-ring $S$ is called a \textbf{semi-ring with unity} if $S \in 2^\Omega (\leftrightarrow \forall A \in S, A \in \Omega) $ and $\Omega \in S$ for some set $\Omega$. $\Omega$ is called the \textbf{unity} of $S$. Indeed, $\Omega \cap A = A, \forall A \in S$.
\end{definition}

\begin{example}
    \begin{enumerate}
        \item (A semi-ring with unity)
        
        The semi-ring of all semi-open cells in $\R^n$ (To be more precise, we need to add the element $\R^n$ into it. For convenience, we won't clarify this much in the future. The reader should always keep this unity in mind.) is a semi-ring with unity $\R^n$.
        \item (A semi-ring WITHOUT a unity)
        
        The semi-ring of all finite semi-open cells in $\R^n$: NO unity ($\R^n$)!
    \end{enumerate}
\end{example}

\begin{definition}
  (Ring of Sets)

  A system of sets $\mathcal{R}$ is called a \textbf{ring} if it satisfies the following two axioms:
  \begin{enumerate}
    \item $\forall A, B \in \mathcal{R}$, $A \cap B \in \mathcal{R}$.
    \item $\forall A, B \in \mathcal{R}$, $A \triangle B = (A \setminus B) \cup (B \setminus A) \in \mathcal{R}$.
  \end{enumerate}
\end{definition}

\begin{remark}
  In fact, a ring $R$ is closed under set difference and finite unions.
  \begin{enumerate}
    \item $\forall A, B \in R$, $A \setminus B = A \triangle (A \cap B) \in R$.
    \item $\forall A, B \in R$, $A \cup B = (A \triangle B) \triangle (A \cap B) \in R$.
  \end{enumerate}

  Conversely, we can derive being closed under set intersection and symmetric difference based on being closed under set difference and finite unions as follows:
  \begin{enumerate}
    \item $\forall A, B \in R$, $A \cap B =  ((A \cup B) \setminus (A \setminus B)) \setminus (B \setminus A)\in R$.
    \item $\forall A, B \in R$, $A \triangle B = (A \cup B) \setminus (A \cap B) \in R$.
  \end{enumerate}

  As a result, we arrive with the same definition of ring requiring closeness under set difference and finite unions.
\end{remark}

\begin{definition}
  (Ring of Sets - Alternative Definition)

  A system of sets $\mathcal{R}$ is called a \textbf{ring} if it satisfies the following two axioms:
  \begin{enumerate}
    \item $\forall A, B \in \mathcal{R}$, $A \setminus B \in \mathcal{R}$.
    \item $\forall A, B \in \mathcal{R}$, $A \cup B \in \mathcal{R}$.
  \end{enumerate}
\end{definition}

\begin{example}
  (A semi-ring but NOT a ring)

  The semi-ring of all cells in $\R^n$: not ensuring the closeness under union!
\end{example}

\begin{definition}
    (Algebra)
    
    A ring with unity is called an \textbf{algebra of sets}.
\end{definition}

\begin{example}
  (A ring but NOT an algebra)

  Consider $R = \{ A \subset \N : |A| < +\infty \}$.
  $R$ is a ring, but $\N \notin R$, which means it doesn't have a unity.
\end{example}

\begin{proposition}
  \begin{enumerate}
    \item A ring is a semi-ring.
    \item $\forall$ system of sets $P$, $\exists$ a \textbf{minimal ring} $\mathcal{R}(P) \supset P$.
  \end{enumerate}
  \begin{proof}
    1. Let $\mathcal{R}$ be a ring. Then $\forall A, B \in \mathcal{R}$, $A \setminus B = A \setminus B (!)= A \triangle (A \cap B) \in \mathcal{R}.$

    2. Start with $\mathcal{R}_0 = 2^\Omega$, where $\Omega$ is the union of all sets in $P$. Let $\{R_\alpha\}$ be the collection of all rings containing $P$. Then
    \( \mathcal{R}(P) := \bigcap_\alpha R_\alpha \) is the minimal ring containing $P$ (it is clearly again a ring!).
  \end{proof}
\end{proposition}

\begin{proposition}
  Let $S$ be a semi-ring, then
  $$\mathcal{R}(S) = \{\bigcup_{j=1}^m A_j, A_j \in S, m \in \N: arbitrary\} \Leftrightarrow \{\bigsqcup_{j=1}^s A_j, A_j \in S, s \in \N: arbitrary\}$$
  \begin{proof}
    "$\Leftrightarrow$":

    Firstly, the claimed system $\mathcal{R}(S)$ is indeed a ring.

    $A = \sqcup_{j=1}^S A_j, B = \sqcup_{i = 1}^m B_i, A \cap B = \sqcup_{i,j}(A_j \cap B_i) \in S \subset \mathcal{R}(S)$.

    $\Rightarrow A \triangle B = (A \setminus B) \cap (B \setminus A)  = \sqcup_{j=1}^m \cap_{i=1}^m (A_j \setminus B_i) \in S \subset \mathcal{R}(S)$.

    Thus, $\mathcal{R}(S)$ is a ring.

    Next, $\forall$ other ring $\tilde{\mathcal{R}}(S)$ containing $S$, it must contain all elements of $\mathcal{R}(S)$.

    i.e. $\tilde{\mathcal{R}}(S) \supset \mathcal{R}(S)$ $\Rightarrow \mathcal{R}(S)$ is the minimal ring containing $S$.
  \end{proof}
\end{proposition}

\begin{definition}
  ($\sigma$-algebra)

  A system of sets $\mathcal{A}$ is called a \textbf{$\sigma$-algebra} if 
  
  1. $\mathcal{A} \subset 2^{\Omega}, \Omega \in \mathcal{A}$;
  
  2. $\mathcal{A}$ is an algebra with unity $\Omega$;
  
  3. $\forall A_1, A_2, \ldots$ (finite or infinite family of sets!) with $\forall j: A_j \in \mathcal{A}$ it holds $\cup_{j=1}^{\infty} A_j \in \mathcal{A}$.
\end{definition}

\begin{proposition}
  \begin{enumerate}
    \item A $\sigma$-algebra is closed under taking the implement: $A^c = \Omega \setminus A \in \mathcal{A}$ since a $\sigma$-algebra is a ring with unity $\Omega$. It is closed under set difference.
    \item $\emptyset \in \mathcal{A}$ since $\emptyset = \Omega^c$ or $\emptyset = \Omega \setminus \Omega$.
    \item A $\sigma$-algebra is closed under finite or countable union thanks to its definition and the fact that $\emptyset \in \mathcal{A}$
    \item A $\sigma$-algebra is closed under finite or countable intersection:

      $\forall A_1, A_2, \ldots$ (finite or infinite family of sets!) with $\forall j: A_j \in \mathcal{A}$, we have

      $\cap_{j=1}^{\infty} A_j = \Omega \setminus \cup_{j=1}^{\infty} (\Omega \setminus A_j) \in \mathcal{A}$
    \item A $\sigma$-algebra is closed under countable symmetric difference.
  \end{enumerate}
\end{proposition}

\begin{remark}
  \underline{Question}: What are the minimal conditions we need to define/prove a $\sigma$-algebra?

  \underline{Answer}: I prefer the following three minimal conditions:

  1. Unity: $\Omega \in \mathcal{A}$.

  2. Closed under set difference: If $A \in \mathcal{A}$, then $A^c = (\Omega \setminus A) \in \mathcal{A}$.

  3. $\sigma$-additivity: If $A_1, A_2, \ldots \in \mathcal{A}$, then $\bigcup_{j=1}^\infty A_j \in \mathcal{A}$.
\end{remark}

\begin{proposition}
  $\forall S \in 2^\Omega, \exists !$ minimum $\sigma$-algebra $\mathcal{A}(S) \supset S$.
  \begin{proof}
    Similar as the proof for $\mathcal{R}(S)$.
  \end{proof}
\end{proposition}

\begin{remark}
  \underline{Upshot 1}:

  In general:

  System of sets$\Rightarrow$ Semi-ring $\Rightarrow$ Ring $\Rightarrow$ Algebra with unity $\Rightarrow \sigma$-Algebra.

  Now, start with a semi-ring with unity $S$

  $\rightarrow$ could generate a ring $\mathcal{R}(S)$ (still equipped with a unity $\Omega$)

  $\rightarrow$ A ring with unity is actually an algebra with unity!

  $\rightarrow$ An algebra of sets: $\mathcal{A}(R(S)) = \mathcal{A}(S)$.
\end{remark}

\begin{remark}
  \underline{Upshot 2}:

  A system of sets $S$

  $\rightarrow$ ensuring the two axioms: closeness under intersection and being able to be decomposed into some disjoint subsets

  $\rightarrow$ A semi-ring!

  $\rightarrow$ could generate a ring $\mathcal{R}(S)$!

  $\rightarrow$ A ring which satisfies closeness under: (intersection and symmetric difference) or (union and difference)

  $\rightarrow$ equip with a unity

  $\rightarrow$ An algebra of sets!
\end{remark}

\begin{definition}
  (Borel $\sigma$-algebra)

  The \textbf{Borel $\sigma$-algebra} on $\R^n$ is defined as the minimum $\sigma$-algebra containing all open sets in $\R^n$, denoted as $\mathcal{B}(\R^n)$.
\end{definition}

\begin{remark}
  Note that $\mathcal{B}(\R^d)$ also contains all closed sets in $\R^d$ since it is closed under difference (open $\rightarrow$ semi-open $\rightarrow$ closed).

  Thus, an alternate definition of $\mathcal{B}(\R^d)$ is the minimum $\sigma$-algebra containing all closed sets in $\R^d$.
\end{remark}

\begin{remark}
  In advanced probability theory, we focus on such Borel $\sigma$-algebra to study all possible events. One can find more in \textit{Foundations of the Theory of Prabability} by A.N.Kolomogorov.
\end{remark}

\clearpage


\subsection{Measure, Measure Space}

\subsubsection{Measure}

\begin{definition}
  (Measure on a Semi-Ring)

  Let $S$ be a semi-ring. A function $\mu: S \rightarrow [0, + \infty)$ is called a \textbf{(finitely additive) measure on $S$} if it satisfies the following two axioms:
  \begin{enumerate}
    \item (Non-negativity) $\forall A \in S, \mu(A) \geq 0$.
    \item (Finite Additivity) If $A, A_1, A_2, \ldots, A_n \in S$ such that $A = \bigsqcup_{j=1}^n A_j$, then
      \(\mu(A) = \sum_{j=1}^n \mu(A_j).\)
  \end{enumerate}
\end{definition}

\begin{proposition}
  \begin{enumerate}
    \item $\mu(\emptyset) = 0$.
    \item $\forall A, B \in S, A \subset B$, we have $\mu(A) \leq \mu(B)$.
  \end{enumerate}
  \begin{proof}
    \begin{enumerate}
      \item $\emptyset = \emptyset \cup \emptyset \Rightarrow \mu(\emptyset) = 2\mu(\emptyset)$.
      \item Since $S$ is a semi-ring, there exist $A_1, A_2, \ldots, A_m \in S$, s.t. $B \setminus A = \bigsqcup_{l=1}^p A_j$

        $\Rightarrow B = A \bigsqcup (\bigsqcup_{j=1}^p A_j) \Rightarrow \mu(B) = \mu(A) + \Sigma_{j=1}^p \mu(A_j) \geq \mu(A)$.
    \end{enumerate}
  \end{proof}
\end{proposition}

\begin{example}
  On the semi-ring $\{$all finite semi-open cells in $\R^n\}$, we define a measure as follows:

  A finite semi-open cell $C = I_1 \times I_2 \times \ldots \times I_n$ in $\R^n$, define $\mu(C) := l(I_1) \times l(I_2) \times \ldots \times l(I_n)$, where $l(I) := $length of $I$ and we are measuring the cell's "volume".

  Such $\mu$ is called the \textbf{Lebesgue measure on all finite semi-open cells in $\R^n$}.
\end{example}

\begin{proposition}
  $\forall$ measure on a semi-ring $S$ can be extended (with identical proerties) to $R(S)$.
  \begin{proof}
    For $A = \sqcup_{j=1}^m A_j \in \mathcal{R}(S)$ with $A_j \in \mathcal{R}(S)$, define $\mu(A) := \Sigma_{j=1}^m \mu(A_j)$. (We need to firstly deal with $A_j \in S$, and then gradually scan the whole $\mathcal{R}(S)$ based on measure-already-defined sets.)

    \underline{Well-defined (Correctness)}: Suppose $A = \sqcup_{j=1}^p A_j = \sqcup_{i=1}^s A_i^{\prime}$. We have

    $\Sigma_{j=1}^p \mu(A_j) = \{$using the finite additivity of $\mu$, and $A_j = A_j \cap A = \sqcup_{i=1}^s (A_j \cap A_i^{\prime})\}$ $= \Sigma_{j=1}^p (\Sigma_{i=1}^s \mu(A_j \cap A_i^{\prime})) = \Sigma_{i=1}^s (\Sigma_{j=1}^p \mu(A_i^{\prime} \cap A_j)) = \Sigma_{i=1}^s \mu(A_i^{\prime})$. $\checkmark$

    \underline{Non-negativity}: Clearly, $\mu(A) \geq 0$. $\checkmark$

    \underline{Finite Additivity}: Suppose $A, B \in R(S): A \cap B = \emptyset$.
    $A = \sqcup_{j=1}^p A_j, B = \sqcup_{i=1}^q B_j$, with $A_j, B_i \in S$.

    $\Rightarrow A \sqcup B = (\sqcup_{j=1}^p A_j) \sqcup (\sqcup_{i=1}^q B_i)$

    $\Rightarrow \mu(A \sqcup B) = \Sigma_{j=1}^p \mu(A_j) + \Sigma_{i=1}^q \mu(B_i)$

    Same for finite union of sets. $\checkmark$
  \end{proof}
\end{proposition}

\begin{proposition}
  (Proerties of a Measure on a ring $\mathcal{R}$)

  \begin{enumerate}
    \item $\mu(\emptyset) = 0$.
    \item If $A, B \in R, A \subset B$, then $\mu(A) \leq \mu(B)$.
    \item (\textbf{Semi-Additivity}) If $A \subset \cup_{j=1}^n A_j$, with $A, A_j \in R$, then $\mu(A) \leq \Sigma_{j=1}^n \mu(A_j)$.

      Now, switch from $\bigcup_{j=1}^n$ to $\bigsqcup_{j=1}^n$:

      Set $A_1^{\prime} := A_1, A_2^{\prime} := A_2 \setminus A_1, A_3^{\prime} := A_3 \setminus \cup_{j=1}^2 A_j, \ldots$

      Now, we have $\bigcup_{j=1}^n A_j = \bigsqcup_{j=1}^n A_j^{\prime}$.

      Thus, $A \subset \bigsqcup_{j=1}^n A_j^{\prime}$ (even more: $A = (\bigsqcup_{j=1}^n A_j^{\prime}) \bigcap A = \bigsqcup_{j=1}^n (A_j^{\prime} \bigcap A)$ !).

      Then, \(\mu(A) = \bigsqcup_{j=1}^n \mu(A_j^{\prime} \bigcap A) \leq \Sigma_{j=1}^n \mu(A_j^{\prime}) \leq \Sigma_{j=1}^n \mu(A_j).\)
  \end{enumerate}
\end{proposition}

\begin{remark}
  \underline{Question}: Could prop. 5.30 (3) maintain for a measure on a semi-ring? Why?

  \underline{Answer}: NO!!! The key difference between a semi-ring and a ring is that: in a semi-ring $S$, the diffence between sets may not belong to $S$, which means though they could be represented as disjoint unions of sets in $S$, they do NOT have measure defined on them! Then the inequality chain cannot go forward anymore.
\end{remark}

\begin{remark}
  \underline{Upshot}: What we have done so far:

  On a semi-ring $S$: we can define a finite-additive measure

  $\rightarrow$ extend to the whole ring generated by $S$: $\mathcal{R}(S)$
\end{remark}


\subsubsection{$\sigma$-Additivive Measure}

\begin{definition}
  ($\sigma$-additivity)
    
  A measure $\mu$ on a semi-ring $S$ is called to satisfy \textbf{$\sigma$-additivity} \textbf{(countable-additivity)} if for any $A \in S$, $\{A_j\}_{j=1}^{\infty} \subset S$ such that $A = \bigsqcup_{j=1}^\infty A_j$, we have
  \(\mu(A) = \sum_{j=1}^\infty \mu(A_j).\)
\end{definition}

\begin{remark}
  \textbf{\underline{Warning}}: A $\sigma$-algebra is not necessarily $\sigma$-additive! 
    
  Also note that $\sigma$-additivity always implies \textbf{semi-$\sigma$-additivity} (sometimes also called \textbf{subadditivity}): 
    
  $\forall A \subset \bigcup_{j=1}^{\infty} A_j, A,A_j \in S, \mu(A) \leq \Sigma_{j=1}^{\infty}\mu(A_j)$.

  And more importantly, finite additivity implies semi-$\sigma$-additivity also!
\end{remark}

\begin{example}
  \begin{enumerate}
    \item Let $\Omega = \N$, $S = 2^\Omega$. Define $\mu(A) := \Sigma_{j \in A} p_j$, where $p_j$ is the ''weight'' assigned to element $j \in \N$ satisfying $\Sigma_{j=1}^\infty p_j = 1$ (or any finite number). Then $\mu$ is a $\sigma$-additive measure on $S$.
    \item Let $\Omega = \N$, $S = 2^\Omega$. Define $\mu(A) := |A|$ (if $A$ is infinite, $\mu(A) := +\infty$). Then $\mu$ is a $\sigma$-additive measure on $S$. (View "weight" being $1$ for all elements. This is the case violating the requirement "$\Sigma_{j=1}^\infty p_j =$ any finite number" in example 1.)
    \item (Lebesgue measure on all finite semi-open cells in $R^n$)

      Let $S = \{$all finite semi-open cells in $\R^n \}$. We know that $S$ is a semi-ring.

      $\mu(C) := l(I_1) \times l(I_2) \times \ldots \times l(I_n)$, where $l(I) := $length of $I$.

      Then $\mu$ is a $\sigma$-additive measure on $S$.
      \begin{proof}
        We already know that $\mu$ is a measure on the semi-ring $S$. $\mu$ is finitely additive.

        Suppose $A \in S, \{A_j\}_{j=1}^{\infty} \in S, A = \sqcup_{j=1}^{\infty} A_j$.

        WTS: $\mu(A) = \Sigma_{j=1}^{\infty} A_j$

        \underline{Step 1}: $\forall n \in \N, A \supset \sqcup_{j=1}^n A_j$

        $\Rightarrow \Sigma_{j=1}^n \mu(A_j) = \{finit-additivity\} =\mu(\sqcup_{j=1}^n A_j) \leq \mu(A)$

        $\Rightarrow$ Take limit $n \rightarrow \infty$, we have $\Sigma_{j=1}^{\infty} \mu(A_j) \leq \mu(A)$. $\checkmark$

        \underline{Step 2}: Let $A = [\alpha_1,\beta_1) \times \cdots \times [\alpha_n,\beta_n)$ be a finite semi-open cell in $\mathbb{R}^n$, and suppose $A = \bigsqcup_{j=1}^{\infty} A_j,$ where each $A_j$ is also a semi-open cell, and the $A_j$'s are pairwise disjoint.
        
        \underline{Step 2.1}: Partition of $A$ into uniform subcells.
        
        For each integer $m \ge 1$, divide each coordinate interval $[\alpha_i,\beta_i)$ into $m$ equal subintervals: $I^{(m)}_{i,k_i} = \big[\alpha_i + k_i(\beta_i-\alpha_i)/m,\; \alpha_i + (k_i+1)(\beta_i-\alpha_i)/m\big), \qquad k_i = 0,1,\dots,m-1$.
        
        Define the finite family of subcells $\mathcal{Q}_m = \Big\{ Q_{k}^{(m)} = I^{(m)}_{1,k_1} \times \cdots \times I^{(m)}_{n,k_n} : 0 \le k_i \le m-1 \Big\}$.
        
        Then the cells in $\mathcal{Q}_m$ are pairwise disjoint and satisfy $A = \bigsqcup_{Q \in \mathcal{Q}_m} Q$.

        In fact, $|\mathcal{Q}_m| = m^n$, which is finite. By finite additivity of $\mu$, $\mu(A) = \sum_{Q \in \mathcal{Q}_m} \mu(Q)$.
        
        \underline{Step 2.2}: Classification of subcells.
        
        For each $Q \in \mathcal{Q}_m$, there are two possibilities:
        
        1. $Q \subset A_j$ for some $j$;
        
        2. $Q$ intersects at least two distinct sets $A_{j_1}, A_{j_2}$.
        
        Let $\mathcal{Q}_m^{(1)} = \{ Q \in \mathcal{Q}_m : \exists j,\; Q \subset A_j \}, \mathcal{Q}_m^{(2)} = \mathcal{Q}_m \setminus \mathcal{Q}_m^{(1)}$.
        
        Define $A_m^{(1)} = \bigcup_{Q \in \mathcal{Q}_m^{(1)}} Q, A_m^{(2)} = \bigcup_{Q \in \mathcal{Q}_m^{(2)}} Q$.
        
        Then $A = A_m^{(1)} \bigsqcup A_m^{(2)}$, and by finite additivity, $\mu(A) = \mu(A_m^{(1)}) + \mu(A_m^{(2)})$.
        
        \underline{Step 2.3}: Estimate of $\mu(A_m^{(1)})$.
        
        Since every $Q \in \mathcal{Q}_m^{(1)}$ is contained in some $A_j$, and all $Q$’s are disjoint,
        $\mu(A_m^{(1)}) = \sum_{Q \in \mathcal{Q}_m^{(1)}} \mu(Q) \le \sum_{j=1}^{\infty} \mu(A_j)$.
        
        \underline{Step 2.4}: Estimate of $\mu(A_m^{(2)})$.
        
        Each $Q \in \mathcal{Q}_m^{(2)}$ intersects at least two distinct cells $A_{j_1}, A_{j_2}$. Thus, every such $Q$ intersects the boundary of some $A_j$.
        
        Denote $\Gamma = \bigcup_{j=1}^{\infty} \partial A_j$. Each $\partial A_j$ is contained in a finite union of $(n-1)$–dimensional hyperrectangles parallel to the coordinate axes; hence $\Gamma$ is a countable union of such hyperrectangles. Therefore, $\mu(\Gamma) = 0$.
        
        Let $\delta_m = \max_i \frac{\beta_i-\alpha_i}{m}$ be the mesh size of the partition $\mathcal{Q}_m$. Then $A_m^{(2)}$ is contained in the $\delta_m$–neighborhood of $\Gamma$ inside $A$. Because $\Gamma$ has measure zero, for any $\varepsilon > 0$ there exists $\eta > 0$ such that the $\eta$–neighborhood of $\Gamma$ has $\mu$–measure less than $\varepsilon$. For all sufficiently large $m$ (namely $m > (\max_i (\beta_i-\alpha_i))/\eta$), we have $\delta_m < \eta$ and hence $\mu(A_m^{(2)}) < \varepsilon$. This shows $\lim_{m \to \infty} \mu(A_m^{(2)}) = 0$.
        
        Combining above, $\mu(A) = \mu(A_m^{(1)}) + \mu(A_m^{(2)}) le \sum_{j=1}^{\infty} \mu(A_j) + \mu(A_m^{(2)})$, and letting $m \to \infty$ gives $\mu(A) \le \sum_{j=1}^{\infty} \mu(A_j)$. $\checkmark$
      \end{proof}
    \item (Finitely Additivive BUT NOT $\sigma$-Additive) 
    
    Let $\Omega = (0,1) \cap \mathbb{Q}.$
    Define the collection $\mathcal{R} = \{A \subset \Omega : A \text{ is finite or co-finite in } \Omega\},$ where “co-finite” means that $\Omega \setminus A$ is finite. Then $\mathcal{R}$ is a ring, since the family of all finite or co-finite subsets of any countable set is closed under finite unions and differences.
    
    Define $\mu : \mathcal{R} \to [0,\infty)$ by $\mu(A) = 0$, if $A$ is finite; $1$, if $A$ is co-finite in $\Omega$.
    
    We verify that $\mu$ is finitely additive.
    
    If $A,B\in\mathcal{R}$ are disjoint, then:

	1.	If both $A$ and $B$ are finite, $A\cup B$ is finite, so $\mu(A\cup B)=0=\mu(A)+\mu(B).$

	2.	If one is finite and the other co-finite, their union is co-finite, so $\mu(A\cup B)=1=\mu(A)+\mu(B).$

	3.	It is impossible for two disjoint co-finite subsets to exist in $\Omega$, so no contradiction arises.
    
    Hence $\mu$ is finitely additive.
    
    Now enumerate $\Omega = \{q_1,q_2,q_3,\dots\}$ and set $A_j = \{q_j\}$.
    
    Then each $A_j$ is finite, hence $\mu(A_j)=0$. Also note that $\Omega = \bigsqcup_{j=1}^{\infty} A_j$.
    
    If $\mu$ were $\sigma$-additive, we would have $\mu(\Omega) = \sum_{j=1}^{\infty} \mu(A_j) = 0$.
    But by definition $\mu(\Omega)=1$. Therefore $\mu$ FAILS to be $\sigma$-additive, even though it is finitely additive.
  \end{enumerate}
\end{example}

\begin{remark}
  A measure $\mu$ with $\sigma$-additivity on $S$ could extend to a measure with $\sigma$-additivity on $\mathcal{R}(S)$ by defining $\mu\left( \bigsqcup_{j=1}^m A_j \right) := \Sigma_{j=1}^m \mu(A_j)$, with $A_j \in S$: disjoint.

  While $\sigma$-additivity of $\mu$ on $\mathcal{R}(S)$ can be derived from $\sigma$-additivity on $S$, note that we still have the weaker condition satisfied: \textbf{semi-$\sigma$-additivity}, i.e. $\forall A \subset \cup_{j=1}^\infty A_j, A, A_j \in \mathcal{R}(S), \mu(A) \leq \Sigma_{j=1}^\infty \mu(A_j)$.
\end{remark}


\subsubsection{Outer Lebesgue Measure}

\underline{Setting}: $S$ - semi-ring with unity $\Omega$; $\mu$ - $\sigma$-additive measure on $S$; $\mathcal{R}(S) = \mathcal{A}(S)$ - the minimum algebra containing $S$.

\begin{definition}
    (outer Lebesgue measure of a set E)
    
    Let $\mu$ be a $\sigma$-additive measure on a semi-ring $S$ with unity $\Omega$ (so, $S \subset 2^\Omega$). 
    
    For any $E \subset \Omega$, define
    
    \[\mu^*(E) := \inf \left\{ \Sigma_{j=1}^\infty \mu(A_j) : A_j \in S, A \subset \cup_{j=1}^\infty A_j \right\}.\]
    
    Then, $\mu^*$ is called the \textbf{outer(exterior) Lebesgue measure of a set E} induced by $\mu$ on $\Omega$.
\end{definition}

\begin{remark}
  The outer measure is to define the measure on sets outside of $S$ beased on the \textbf{pre-measure} on $S$.

  The outer measure $\mu^*$ of a set $E$ \textbf{always exists} (may be infinitely many), since

  1. $\left\{ \Sigma_{j=1}^\infty \mu(A_j) : A_j \in S, A \subset \cup_{j=1}^\infty A_j \right\}$ at least contains $\Omega$;

  2. Consider the real numbers in $\left\{ \Sigma_{j=1}^\infty \mu(A_j) : A_j \in S, A \subset \cup_{j=1}^\infty A_j \right\}$, they have lower bound $0$. By the completeness of $\R$, the infimum exists.

  \textbf{\underline{Warning}}: In general, one CANNOT claim that $\mathcal{A}(S) \supset \mathcal{A}(\Omega)$. This is also the key problem of out outer measure being not able to capture all the information in the algebra generated by $\Omega$!
\end{remark}

\begin{example}
    (An invisible set under the outer measure)
    
    Let $S = \{ [a,b) : a,b \in \mathbb{Q},\, a < b \}$ ($S$ is indeed a semi-ring with unity), and define the pre-measure $\mu([a,b)) = b - a$. The outer measure $\mu^*$ on $2^{\mathbb{R}}$ is defined by $\mu^*(E) = \inf \{ \sum_{j=1}^{\infty} \mu(A_j) : A_j \in S,\, E \subseteq \bigcup_{j=1}^{\infty} A_j \}$.
    
    Consider the set $E = \mathbb{Q} \cap [0,1)$. We will show that $\mu^*(E) = 1$, while $\mu^*(\{q\}) = 0$ for all $q \in E$. Hence, $\mu^*(\bigsqcup_{q\in E}\{q\}) = 1 > 0 = \sum_{q\in E}\mu^*(\{q\})$, which demonstrates that $\mu^*$ is not countably additive, even for disjoint sets.
\end{example}

\begin{remark}
This example shows that $\mu^*$ cannot "see" the internal structure of sets outside the algebra $\mathcal{A}(S)$ (But we are still in $\mathcal{A}(\Omega)$!). Although $E$ is a countable, measure-zero set in the intuitive sense, any cover of $E$ by rational half-open intervals must in fact cover the entire interval $[0,1)$. Hence, the outer measure treats $E$ as if it were as large as $[0,1)$.

\underline{A simple point of view}: We know that there is quite possible to find a set $E$ in $\mathcal{A}(\Omega) \setminus \mathcal{A}(S)$. For such set, we cannot find a quite precise covering of it, so we can only use the whole unity $\Omega$ as a part of our approximation.
\end{remark}

\begin{remark}
    (The philosophy behind outer measure)
    
    Why do we call it an "outer measure"? 
    
    The name comes from its construction principle: we measure a set \emph{from the outside}. Given a subset $E \subseteq \Omega$, we generally cannot measure $E$ directly, because $E$ may be too irregular or may not belong to the algebra $\mathcal{A}(S)$ where the original measure $\mu$ is defined. Instead, we approximate $E$ by sets $A_j \in S$ that cover $E$ from the outside and take the smallest possible total measure among all such coverings. 
    
    Formally, $\mu^*(E) = \inf\{\sum_j \mu(A_j) : E \subseteq \bigcup_j A_j,\, A_j \in S\}$, which expresses the idea of an \emph{outer approximation}. The measure does not come from the intrinsic structure of $E$, but from the minimal "outer shell" built using measurable sets in $S$. 
    
    Philosophically, $\mu^*$ represents the best information we can obtain about the size of $E$ given our limited "vocabulary" $S$. It is an act of estimation under partial visibility: we look at $E$ through a coarse geometric lens and ask, "How small can the total measure of the covering be if I only use shapes I can measure?"
    
    Thus, it is called an \emph{outer measure} because it always measures from the \emph{outside}, enclosing $E$ within measurable sets rather than dissecting it from the inside.
\end{remark}

\begin{proposition}
  \begin{enumerate}
    \item $\mu^*$ always $\exists$, and $\mu^*(A) \geq 0, \forall A \subset \Omega$.
    \item We can equivalently say in the definition of $\mu^*$ that $A_j$ are disjoint.
    \item $\forall A \in \mathcal{A}(S)$, $\mu(A) = \mu^*(A)$
      \begin{proof}
        On one hand, by the semi-$\sigma$-additivity, $\mu(A) \leq \Sigma_{j=1}^{\infty} \mu(A_j)$ if $\cup_{j=1}^{\infty} A_j \supset A$.

        $\Rightarrow$ Take $inf$: $\mu(A) \leq \mu^*(A)$;

        On the other hand, take the trivial covering: $A_1 = A$,
        
        $\mu(A) = \mu(A_1) = \mu(A_1 \bigsqcup_{j=1}^{\infty}\emptyset) \geq \mu^*(A)$,

        $\Rightarrow \mu(A) = \mu^*(A)$.
      \end{proof}
    \item If $E_1 \subset E_2 \subset \Omega$, then $\mu^*(E_1) \leq \mu^*(E_2)$ (since any covering of $E_2$ is also a covering of $E_1$).
    \item (\emph{Semi-$\sigma$-additivity of $\mu^*$}) 
    
    If $E \subset \cup_{j=1}^\infty E_j$, $E,E_j \subset \Omega$, then $\mu^*(E) \leq \Sigma_{j=1}^\infty \mu^*(E_j)$. (this CANNOT be improved even if $E = \sqcup_{j=1}^{\infty} E_j$ --- check our warning above!)
      \begin{proof}
        $\forall \eps > 0$, 
        
        $\forall j$, choose $\{A_{j,k}\}_{k=1}^{\infty} \subset S$ such that $E_j \subset \cup_{k=1}^{\infty} A_{j,k}$ and

        \(\Sigma_{k=1}^{\infty} \mu(A_{j,k}) \leq \mu^*(E_j) + \frac{\eps}{2^j}\) (thanks to the infimum property).

        Thus, \(E \subset \cup_{j=1}^{\infty} E_j \subset \cup_{j=1}^{\infty} \cup_{k=1}^{\infty} A_{j,k}.\)

        Thus, by the definition of $\mu^*$ and semi-$\sigma$-additivity of $\mu$,

        \(\mu^*(E) \leq \Sigma_{j=1}^{\infty} \Sigma_{k=1}^{\infty} \mu(A_{j,k}) \leq \Sigma_{j=1}^{\infty} \left( \mu^*(E_j) + \frac{\eps}{2^j} \right) = \Sigma_{j=1}^{\infty} \mu^*(E_j) + \eps.\)

        Let $\eps \rightarrow 0^+$, we get the desired result.
      \end{proof}
  \end{enumerate}
\end{proposition}

\begin{example}
  Let's fix a bounded cell $\Omega$ in $\R^d$. Let $S = \{$all cells $C \subset \Omega\}$.

  Define $\mu(\{p\}) = 0$ for all $p \in \Omega$. Consider $E = \Omega \cap \Q^n, E = \{q_1, q_2, \ldots\}$

  $\Rightarrow \mu^*(E) \leq \Sigma_{j=1}^{\infty} \mu^*(\{q_j\}) = \Sigma_{j=1}^{\infty} \mu(\{q_j\}) = 0 \Rightarrow \mu^*(E) = 0$.

  $\mu^*(\Omega \setminus E) \leq \mu^*(\Omega) = \mu(\Omega)$

  But by semi-$\sigma$-additivity,
  \(\mu(\Omega) = \mu^*(\Omega) \leq \mu^*(E) + \mu^*(\Omega \setminus E) = \mu^*(\Omega \setminus E).\)

  $\Rightarrow \mu^*(\Omega \setminus E) = \mu(\Omega)$, which means that the outer measure CANNOT distinguish the counterble but sparce set $\Q^n$.
\end{example}

With such outer measure, one can similarly get:

\begin{definition}
  (Inner Lebesgue Measure of a Set E)

  Let $\mu$ be a $\sigma$-additive measure on a semi-ring $S$ with unity $\Omega$ (so, $S \subset 2^\Omega$).

  Based on the outer measure $\mu^*$, for any $E \subset \Omega$, define

  \[\mu_*(E) = 1 - \mu^*(\Omega \setminus E)\]

  Then, $\mu_*$ is called the \textbf{inner(interior) Lebesgue measure of a set E} induced by $\mu$ on $\Omega$.
\end{definition}


\begin{proposition}
  $\forall E \subset \Omega, \mu_*(E) \leq \mu^*(E)$
\end{proposition}


\subsubsection{Lebesgue Extension of a $\sigma$-Additive Measure on $\mathcal{A}(S)$}

\begin{definition}
  (Lebesgue Measurable)

  Let $S$ be a semi-ring with unity $\Omega$, and $\mu$ be a $\sigma$-additive measure on $S$. $R(S) = \mathcal{A}(S)$ --- the minimum algebra containing $S$, $\mathcal{A}(S) \subset 2^\Omega$. 
  
  A set $E \subset \Omega$ is called \textbf{(Lebesgue) measurable} if and only if $\forall \eps > 0$, $\exists B_{\eps} \in \mathcal{A}(S)$ such that $\mu^*(E \triangle B_{\eps}) = \mu^*(E \setminus B_{\eps}) + \mu^*(B_{\eps} \setminus E) < \eps$, i.e. the set $E$ can be approximated by a set $B_{\eps} \in \mathcal{A}(S)$. We call such condition the \textbf{approximation property (or being measurable in the sense of Lebesgue)}.
\end{definition}

\begin{example}
  In this setting, let $\mu^*(E) = 0$, then $E$ is measurable: Choose $B_{\eps} = \emptyset$, then \(\mu^*(E \triangle B_{\eps}) = \mu^*(E) < \eps\).
\end{example}

\begin{definition}
  (Lebesgue Measurable: Altanative Definition)

  Let $S$ be a semi-ring with unity $\Omega$, and $\mu$ be a $\sigma$-additive measure on $S$.
  
  A set $E \subset \Omega$ is called \textbf{(Lebesgue) measurable} if and only if $\forall A \subset \Omega$, $\mu^*(E) = \mu^*(E \bigcap A) + \mu^*(E \setminus A)$. Such condition is called to be satisfying the \textbf{Carathéodory\footnote{For a biography of Constantin Carathéodory, please see Appendix A.2.} criterion (or being measurable in the sense of Carathéodory)}.
\end{definition}

\begin{theorem}
  The two definitions above are equivalent.
  \begin{proof}
    Let $S$ be a semi-ring with unity $\Omega$, let $\mu_0$ be a $\sigma$-additive premeasure on $S$ (we emphasizes that this measure is the pre-measure), and let $\mu^*$ be the outer measure obtained from $S$ by the usual covering construction. 
    
    \(\mathcal M:=\Big\{E\subset\Omega:\ \forall X\subset\Omega,\ \mu^*(X)=\mu^*(X\cap E)+\mu^*(X\setminus E)\Big\}\) is the Carath\'eodory $\sigma$-algebra.
    
    \noindent\textbf{Auxiliary facts}:

    (1) For all $X,E,B\subset\Omega$, we have
    
    $\big|\mu^*(X\cap E)-\mu^*(X\cap B)\big|\le \mu^*(E\triangle B),\qquad \big|\mu^*(X\setminus E)-\mu^*(X\setminus B)\big|\le \mu^*(E\triangle B)$, 
    
    which follows from monotonicity and subadditivity of $\mu^*$ (e.g.\ $X\cap E\subset (X\cap B)\cup(E\triangle B)$).
    
    (2) $A(S)\subset\mathcal M$: 
    
    First check $S\subset\mathcal M$ by the additivity of $\mu_0$ on $S$ and the definition of $\mu^*$; since $\mathcal M$ is a $\sigma$-algebra, it contains the algebra $A(S)$.
    
    \noindent\textbf{(Approximation $\Rightarrow$ Carath\'eodory).}Assume $E\subset\Omega$ satisfies: for every $\varepsilon>0$ there is $B_\varepsilon\in A(S)$ with $\mu^*(E\triangle B_\varepsilon)<\varepsilon$.
    
    Fix $X\subset\Omega$. Because $B_\varepsilon\in A(S)\subset\mathcal M$, \(\mu^*(X)\ge \mu^*(X\cap B_\varepsilon)+\mu^*(X\setminus B_\varepsilon)\).
    
    Applying the first auxiliary fact with $B=B_\varepsilon$ gives \(\mu^*(X)\ \ge\ \mu^*(X\cap E)+\mu^*(X\setminus E)-2\,\mu^*(E\triangle B_\varepsilon)\).
    
    Letting $\varepsilon\downarrow0$ yields $\mu^*(X)\ge \mu^*(X\cap E)+\mu^*(X\setminus E)$. The reverse inequality is the subadditivity of $\mu^*$, hence equality holds for all $X$, i.e.\ $E\in\mathcal M$.
    
    \noindent\textbf{(Carath\'eodory $\Rightarrow$ Approximation).}Assume $E\in\mathcal M$. Let $\varepsilon>0$.
    
    By the definition of $\mu^*$ choose a cover $E\subset\bigcup_{k\ge1} S_k$ with $S_k\in S$ such that \(\sum_{k=1}^\infty \mu_0(S_k)\le \mu^*(E)+\varepsilon/3.\)
    
    Write $U_N:=\bigcup_{k=1}^N S_k\in A(S)$ and $U:=\bigcup_{k\ge1} S_k$.
    
    Then $\mu^*(U)\le \mu^*(E)+\varepsilon/3$.
    
    Since $E$ is Carath\'eodory measurable and $E\subset U$, \(\mu^*(U)=\mu^*(E)+\mu^*(U\setminus E)\quad$
    
    $\Rightarrow\quad \mu^*(U\setminus E)\le \varepsilon/3.\)
    
    By semi-$\sigma$-additivity on the tail, choose $N$ so large that $\mu^*(U\setminus U_N)\le \varepsilon/3$.
    
    Hence \(\mu^*(U_N\setminus E)\le \mu^*(U\setminus E)+\mu^*(U\setminus U_N)\le \tfrac{2\varepsilon}{3},\qquad \mu^*(E\setminus U_N)\le \mu^*(U\setminus U_N)\le \tfrac{\varepsilon}{3}\), and therefore $\mu^*(E\triangle U_N)\le \varepsilon$.
    
    With $B_\varepsilon:=U_N\in A(S)$ we obtain the approximation property.
    
    Combining the two implications proves that the two definitions above are equivalent.
  \end{proof}
\end{theorem}

\begin{remark}
  \underline{Think about it}: Can such definition address our problem in the last subsubsection?

  \underline{Answer}: Yes, the Carathéodory criterion directly and completely addresses this problem!
  
  \begin{enumerate}
    \item \underline{It provides a filter:} The definition provides a precise condition to "sieve" the "measurable" sets from the "non-measurable" ones. A set $E$ is declared measurable if and only if it splits \emph{every} other set $A$ in an additive way with respect to the outer measure:
    $\mu^*(A) = \mu^*(A \cap E) + \mu^*(A \setminus E)$

    \item \underline{It constructs the $\sigma$-algebra:} The Carathéodory Extension Theorem (which is based on this definition) proves that the collection $\mathcal{M}$ of all sets $E$ that satisfy this criterion forms a $\sigma$-algebra.

    \item \underline{It guarantees additivity:} The same theorem proves that the outer measure $\mu^*$, when restricted to this $\sigma$-algebra $\mathcal{M}$, becomes a \textbf{countably additive measure}.
\end{enumerate}

In summary, Definition 5.46 is not just an arbitrary definition; it is the precise tool needed to solve the extension problem. It successfully identifies the exact collection of sets ($\mathcal{M}$, the Lebesgue measurable sets) on which the outer measure $\mu^*$ behaves as a true, countably additive measure.

\end{remark}

\begin{remark}
    The definition of a (Lebesgue) measurable set captures the idea of \emph{approximability by “nice” sets}. A set $E \subset \Omega$ is called measurable if it can be arbitrarily well approximated by sets $B_\varepsilon$ from the algebra $\mathcal{A}(S)$, in the sense that the “disagreement region” between $E$ and $B_\varepsilon$, namely the symmetric difference $E \triangle B_\varepsilon$, has arbitrarily small outer measure: $\mu^*(E \triangle B_\varepsilon) < \varepsilon$ for all $\varepsilon > 0$.
    
    Intuitively, this means that even if $E$ itself may be irregular or complicated, we can always find a clean, measurable set $B_\varepsilon$ that almost coincides with $E$ up to an arbitrarily small “error area.” Measurable sets are precisely those whose geometry can be faithfully captured through such approximations.
    
    In the above example, if $\mu^*(E)=0$, then $E$ is trivially measurable. Indeed, we can take $B_\varepsilon = \emptyset$, so that $\mu^*(E \triangle B_\varepsilon) = \mu^*(E) = 0 < \varepsilon$. This illustrates that every \emph{measure-zero set} is measurable: such sets are geometrically “invisible” to the outer measure, since they can be ignored without affecting any measured quantity.
\end{remark}

\underline{Setting}: 

$(\Omega, S, \mu)$ --- $\Omega$ - set, $S$ - semi-ring with unity $\Omega$, $\mu$ - $\sigma$-additive measure on $S$

$\rightarrow$ directly extend to $(\Omega, \mathcal{A}(S), \mu)$, $\mu$: the pre-measure.

$\rightarrow$ introduce $\mu^*$ on the whole $2^\Omega$,

$\rightarrow$ $(\Omega, \mathcal{M}(\Omega), \mu)$, with $\mathcal{M}(\Omega)$: collection of all measurable sets in $\Omega$.

'measurable': $\forall A \in \mathcal{M}(\Omega)$, $\forall \eps > 0$, $\exists B_{\eps} \in \mathcal{A}(S)$ such that $\mu^*(A \triangle B_{\eps}) < \eps.$

\begin{remark}
  To better distinguish $\mu$ and $\mu^*$, for those in the original $\mathcal{A}(S)$, we use $\mu$. Otherwise, we use the notation $\mu^*$. Thus, $*$ emphases that the measure on the set is defined by extanding $\mu$.
\end{remark}

\begin{theorem}
  (Carathéodory's Extension Theorem)

  In the above setting (with pre-measure $\mu$ on $\mathcal{A}(S)$), let $\mathcal{M}(S)$ be the collection of all measurable sets and we set $\mu(A) := \mu^*(A), \forall A \in M(S)$. Then,
  \begin{enumerate}
    \item $\mathcal{M}(S)$ is a $\sigma$-algebra.

      ($M(S)$ extends the original algebra $\mathcal{A}(S)$.)
    \item $\mu^*$ is $\sigma$-additive on $\mathcal{M}(S)$.

      ($M$ extends the original measure $\mu$ on $\mathcal{A}(S)$.)
  \end{enumerate}
  \begin{proof}

    First of all, we know that $\Omega \in \mathcal{M}(\Omega)$.


    \underline{Step I}: prove if $A \in \mathcal{M}(\Omega)$, then $\Omega \setminus A \in \mathcal{M}(\Omega)$.

    Fix $\eps > 0$, $\exists B_{\eps} \in \mathcal{A}(S)$ such that $\mu^*(A \triangle B_{\eps}) < \eps.$ 
    
    Consider $\Omega \setminus B_{\eps} \in \mathcal{A}$. Then, note $(\Omega \setminus A) \triangle (\Omega \setminus B_{\eps}) = A \triangle B_{\eps}$.

    Thus, $\mu^*((\Omega \setminus A) \triangle (\Omega \setminus B_{\eps})) < \eps$ $\Rightarrow \Omega \setminus A \in \mathcal{M}(\Omega)$.

    \underline{Step II}: prove $\forall A_1, A_2, \ldots, A_n \in \mathcal{M}(\Omega)$, we have $\bigcup_{i=1}^n A_i \in \mathcal{M}(\Omega)$.

    Only need to prove for $n=2$ (others by induction).

    $A_1, A_2 \in \mathcal{M}(\Omega)$, $\forall \eps > 0. \exists B_1, B_2 \in \mathcal{A}: \mu^*(A_1 \triangle B_1) < \eps, \mu^*(A_2 \triangle B_2) < \eps$.

    $A = A_1 \bigcup A_2$, we will approximate by $B = B_1 \bigcup B_2$.
    
    Since $(A_1 \bigcup A_2) \triangle (B_1 \bigcup B_2) \subset (A_1 \bigcup B_1) \triangle (A_2 \bigcup B_2)$,

    $\mu^*(A \triangle B) < \mu^*(A_1 \triangle B_1) + \mu^*(A_2 \triangle B_2) < 2\eps$

    $\Rightarrow A_1 \bigcup A_2 \in \mathcal{M}(\Omega)$.

    Thus, the first statement is proved.

    \begin{corollary}
    $\mathcal{M}(\Omega)$ is an algebra.
        
        \begin{proof}
            \begin{itemize}
                \item contains $\Omega$.
                \item closed under taking union: proved above.
                \item closed under intersection:
                \item closed under symmetric difference: $A \triangle B = $
            \end{itemize}
        \end{proof}
    \end{corollary}

    \underline{Step III}: prove $\mu^*$ is finitely additive on $\mathcal{M}(\Omega)$.

    So, $\forall A_1, A_2, \ldots, A_n \in \mathcal{M}(\Omega)$, we need to show $\mu(A_1 \cup A_2 \cup \cdots \cup A_n) = \mu(A_1) + \mu(A_2) + \cdots + \mu(A_n)$.

    Similarly, only need to show for $n=2$.

    Take $A_1, A_2 \in \mathcal{M}(\Omega), A_1 \cap A_2 = \emptyset$.

    $\forall \eps > 0, \exists B_1, B_2 \in \mathcal{A}(S): \mu^*(A_1 \triangle B_1) < \eps, \mu^*(A_2 \triangle B_2) < \eps$.

    Since $B_1 \bigcap B_2 \subset (A_1 \triangle B_1) \bigcup (A_2 \triangle B_2)$, we have $\mu^*$




    \underline{Step IV}: prove $\mu^*$ is a $\sigma$-algebra on $\mathcal{M}(\Omega)$.



    \dots

    \dots


    

    Replace by disjoint union: let $A_1^{\prime} = A_1, A_2^{\prime} = A_2 \setminus A_1, A_3^{\prime} = A_3 \setminus (A_1 \bigcup A_2), \ldots$.

    Then, we have $A = \bigsqcup_{i=1}^{\infty}A_i^{\prime}$.

    We have 







    \underline{Step V}: prove $\mu^*$ is $\sigma$-additive on $\mathcal{M}(\Omega)$.

    WTS: $\forall A_1, A_2, \ldots, A_n \in \mathcal{M}(\Omega)$, we have $\mu^*(A_1 \cup A_2 \cup \cdots \cup A_n) = \mu^*(A_1) + \mu^*(A_2) + \cdots + \mu^*(A_n)$.
    
    


  \end{proof}
\end{theorem}




Conclusion: We end up with a triple ($\Omega, \mathcal{M}(\Omega), \mu$) --- (set: $\Omega$, $\sigma$-algebra: $\mathcal{M}(\Omega)$, $\sigma$-additive measure on $\mathcal{M}(\Omega)$: $\mu$).


\subsubsection{Measure Space}

\begin{definition}
    (Measure Space)

    Such a triple ($\Omega, \mathcal{A}, \mu$) ($\mathcal{A}$ is some $\sigma$-algebra on the set $\Omega$) is called a \textbf{measure space (spcae with measure)}.
\end{definition}

\begin{definition}
    (Completeness of a Measure Space)


    A \textbf{complete} measure (or, more precisely, a complete measure space) is a measure space in which every subset of every null set is measurable (having measure zero).

    More formally, if ($\Omega, \mathcal{A}, \mu$) is a measure space, then it's called \textbf{complete} if and only if $A \subset E \in \mathcal{A}$, $\mu(E) = 0$, $\Rightarrow A \in \mathcal{A}$ (and hence $\mu(A)=0$).
\end{definition}

\begin{example}
    For ($\Omega, \mathcal{M}(\Omega), \mu$), we always have completeness:

    $\mu(A) = 0, E \subset A \Rightarrow 0 \leq \mu^*(E) \leq \mu^*(A) = 0 \Rightarrow E \in \mathcal{M}(\Omega)$.

    But this FAILS in general. For example, $\exists$ measure 0 non-Borel sets, which is contained in some measure 0 Borel sets, so Lebesgure measure $\mu$ on $\R^n$, restricted to Borel $\sigma$-algebra is incomplete.

    However, any incomplete measure space can extend its measure to attain a complete measure space. One just need to follow the Lebesgue extension of a general measure space ($\Omega, \mathcal{A}, \mu$).
\end{example}

\begin{theorem}
    For any measure space $(\Omega, \mathcal{A}, \mu)$, the following holds:
    \begin{enumerate}
        \item $\forall A_1 \subset A_2 \subset A_3 \subset \ldots$ with $A_i \in \mathcal{A}$, $\mu(\bigcup_{i=1}^{\infty} A_i) = \lim\limits_{i \to \infty} \mu(A_i)$.
        \item If $A_1 \supset A_2 \supset A_3 \supset \ldots$, $A_i \in \mathcal{A}$, then $\lim\limits_{i \to \infty} \mu(A_i) = \mu(\bigcap_{i=1}^{\infty} A_i)$.
    \end{enumerate}

    Both 1 and 2 are called the \textbf{continuity} of the measure.

    \begin{proof}
    
    \end{proof}
\end{theorem}

\underline{Question}: What about $\infty$-valued measures?

Consider a space with measure ($\Omega, \mathcal{A}, \mu$), where $\mu$ is a $\bar{\R}$-valued measure. The definition of finite additivity and $\sigma$-additivity is repeated word-by-word:

\textbf{Finite additivity}:
\begin{enumerate}
  \item $\mu(A) \geq 0$;
  \item $\mu(\bigcup_{j=1}^{n} A_j) = \Sigma_{j=1}^{n} \mu(A_j)$;
\end{enumerate}

\textbf{$\sigma$-additivity}:
\begin{enumerate}
  \item $\mu(A) \geq 0$;
  \item $\mu(\bigcup_{j=1}^{\infty} A_j) = \Sigma_{j=1}^{\infty} \mu(A_j)$;
\end{enumerate}

Then we easily deduce several similar properties.

\begin{proposition}
    \begin{enumerate}
        \item $\mu(\emptyset) = 0$.
        \item If $A \subset B,$ then $\mu(A) \leq \mu(B)$.
        \item If $A \subset \bigcup_{j=1}^{\infty} A_j,$ then $\mu(\bigcup_{j=1}^n A_j) \leq \Sigma_{j=1}^{\infty} \mu(A_j)$.
    \end{enumerate}
\end{proposition}

\begin{definition}
    A measure space with $\infty$-valued measure is called \textbf{$\sigma$-finite} if 
    
    $\Omega = \bigsqcup_{k=1}^{\infty} \Omega_k, \Omega_k \in \mathcal{A}$, $\mu(\Omega_k) < \infty$. 
    
    Then $\forall A \subset \mathcal{A}$, $\mu(A) = \Sigma_{k=1}^{\infty} \mu(A \bigcap \Omega_k) =: \Sigma_{k=1}^{\infty} \mu_k(A)$.
\end{definition}

So, essentially, $\mu$ is obtained from $\{\mu_k\}_{k=1}^{\infty}$, with each $\mu_k$ defined on $\mathcal{A}\cap 2^{\Omega_k}$.









\begin{example}
    $\R^n = \bigsqcup_{i_1, i_2, \ldots, i_n}[i_1, i_1+1) \times [i_2, i_2+1) \times \ldots \times [i_n, i_n+1), i_1, \ldots, i_n \in \Z$.

\end{example}










\begin{remark}
  For $\Omega = \bigsqcup_{k=1}^{\infty} \Omega_k = \bigsqcup_{j=1}^{\infty} \Omega_j^{\prime}$

  Mesures agree:





\end{remark}










\clearpage






\subsection{Lebesgue Measure in $\R^n$}

\underline{Goal}: Understand $\mathcal{M}(\R^n)$.

\underline{Main fact}: $\mathcal{B}(\R^n) \subsetneq \mathcal{M}(\R^n) \subsetneq 2^{\R^n}$.










\subsubsection{Basic Intuition for Non-Measurable Sets}

\begin{proposition}
  \begin{enumerate}
    \item (Shift-invariance) If $E_{\alpha} := \{x+\alpha, x\in E, \alpha \in \R^n$: fixed$\}$, then 
    
    $E_{\alpha} \in \R^n \Leftrightarrow E \subset \mathcal{M}(R)$; and we also have $\mu(E_{\alpha} = \mu(E))$.

    It holds since it holds for cells.

    \item $\mathcal{B}(\R^n) \subset \mathcal{M}(\R^n)$
  \end{enumerate}

\end{proposition}

\begin{proposition}
  $\exists$ a non-measurable subset $A \subset [0,1)$.
  \begin{proof}
    On $[0,1)$, consider the following equivalent relation:


  \end{proof}
\end{proposition}





\begin{proposition}
  $\forall A \subset \R$ with $\mu(A)>0$, $A$ contains some $B \subset A$ s.t. $B \notin \mathcal{M}(\R)$.
  \begin{proof}
  
  \end{proof}

\end{proposition}

\begin{remark}
  The same holds in $\R^n$: $\forall A \subset \R^n$ with $\mu(A)>0$, $A$ contains some $B \subset A$ s.t. $B \notin \mathcal{M}(\R^n)$.
\end{remark}






\clearpage



\subsubsection{Cantor Set}

We build a sequence of sets:

$E_0 = [0,1]$

$E_1 = E_0 \setminus I_1$, $I_1 = (\frac{1}{3}, \frac{2}{3})$.

$E_2 = E_1 \setminus I_2$, $I_2 = I_{1,1} \bigcup I_{1,2}$, $I_{1,1} = (\frac{1}{9}, \frac{2}{9})$, $I_{1,2} = (\frac{7}{9}, \frac{8}{9})$.

$E_3 = E_2 \setminus I_3$, $I_3 = I_{2,1} \bigcup I_{2,2} \bigcup I_{2,3} \bigcup I_{2,4}$, $I_{2,1} = (\frac{1}{27}, \frac{2}{27})$, $I_{2,2} = (\frac{4}{27}, \frac{5}{27})$, $I_{2,3} = (\frac{19}{27}, \frac{20}{27})$, $I_{2,4} = (\frac{25}{27}, \frac{26}{27})$.


$\ldots$

We get a sequence of sets $\{E_k\}$, $\forall E_k$ is closed. $\Rightarrow C_0 := \bigcap_{k=1}^{\infty}E_k$, $C_0$ is closed and bounded $\Rightarrow$ $C_0$ is compact.

\begin{definition}
  (Cantor Set)

  Such $C_0$ is called a \textbf{(standard) Cantor set}.
\end{definition}

\begin{proposition}
  \begin{enumerate}
    \item $C_0$ is compact and $C_0 \subset [0,1]$;
    \item $C_0$ is nowhere dense;
    \item $\mu(C_0) = 0$
    \begin{proof}
    
    \end{proof}
    \item $C_0$ is continual.
    \begin{proof}
    
    \end{proof}
  \end{enumerate}
\end{proposition}



\clearpage


\subsubsection{Cantor Staircase Function}

\begin{definition}
  
\end{definition}


\begin{lemma}
  Let $f: \Omega \rightarrow \Omega^{\prime}, S^{\prime} \subset 2^{\Omega^{\prime}}$, then $\mathcal{A}(f^{-1}*(S^{\prime})) = f^{-1}(\mathcal{A}(S^{\prime}))$.
  \begin{proof}
  
  \end{proof}
\end{lemma}

\begin{corollary}
  (Preimage of Borel set is Borel.)

  If $f: [a,b] \rightarrow [c,d]$ is continuous, then $f^{-1}(E^{\prime})$ is Borel, provided $E^{\prime} \subset [c,d]$ is Borel.
  \begin{proof}
    Follows from $f^{-1}(G)$ is open if $G$ is open.
  \end{proof}
\end{corollary}


Now, consider $\phi (x) := x + K(x)$, $\phi : [0,1] \rightarrow [0,2]$, $\phi$ is strictly increasing.




\clearpage

\subsubsection{Construction of a Non-Borel Measurable Set(!)}




\clearpage


\begin{proposition}
  Let $A \subset \R^n$ be a Lebesgue mesurable set, i.e. $A \in \mathcal{M}(\R^n)$.
  
  Then, $\forall \delta > 0$, $\exists$ a closed $F_\delta$ and an open $G_\delta$ satisfying $F_\delta \subset A \subset G_\delta$, s.t. $\mu(A \setminus F_\delta) < \delta$ and $\mu(G_\delta \setminus A) < \delta$.
  \begin{proof}
  
  \end{proof}
\end{proposition}

\begin{corollary}
  $\forall A \in \mathcal{M}(\R^n)$, A can be decomposed as:

  1. $A = F \bigsqcup E$, where $F$ is Borel and $E$ is measure 0;
  
  2. $A = G \setminus E$, where $G$ is Borel and $E \subset G$ is measure 0.
  \begin{proof}
    1. If $\mu(A) < \infty$:
    
    Take $F := \bigcup_{k=1}^{\infty} F_{\frac{1}{2^k}}$.

    Then $F \subset A$ and $\forall k$, $\mu(F) \geq \mu(F_{\frac{1}{2^k}}) > \mu(A) - \frac{1}{2^k}$.

    $\Rightarrow \mu(F) \geq \mu(F) \geq \mu(A)$

    $\Rightarrow \mu(F) = \mu(A)$, $\mu(A \setminus F) = 0$.

    Now, if $\mu(A) = \infty$, then we can write $A = \bigsqcup_{k=1}^{\infty} A_k$ with all $\mu(A_k) < \infty$.

    Then, $\forall A_k = F_k \bigsqcup E_k$.

    Just take $F = \bigsqcup_{k=1}^{\infty} F_k$ and $E = \bigsqcup_{k=1}^{\infty} E_k$.

    2. Proof is analogous to 1.
  \end{proof}
\end{corollary}

\begin{remark}
  Note that the actual $F$ here is an at most countable union of closed sets. And $G$ here is an at most countable intersection of open sets.
\end{remark}

\begin{remark}
  \underline{Reminder}: $\mathcal{A}(f^{-1}(S)) = f^{-1}(\mathcal{A}(S))$.
\end{remark}


\clearpage

\subsection{Measurable Functions}

\subsubsection{What Kind of Functions are Measurable?}

\begin{definition}
  (Measurable Function)

  Let $(\Omega, \mathcal{A}, \mu)$ be a measure space with a complete measure. Then a function $f: \Omega \rightarrow \R$ is called \textbf{measurable} if and only if $\forall$ Borel set $A \subset \R$, it holds $f^{-1}(A) \in \mathcal{M}(\Omega)$.
\end{definition}

\begin{remark}

\end{remark}

\begin{remark}

\end{remark}


\clearpage

\subsubsection{Properties of Measurable Functions}

\begin{theorem}
  \begin{enumerate}
    \item If $f$ is measurable, then $a f + b$ is measurable for $a, b \in \R$;
    
    \noindent\medskip Define $E_c := \{x \in \Omega: af(x)+b<c\}$







    \item If $f$, $g$ are measurable, then the set $\{x: f(x)<g(x)\}$ is measurable.
    \begin{proof}
    
    \end{proof}

    \item Combining \textit{1} and \textit{2}, one can get:
    
    \ldots

    $\Rightarrow f \pm g$ is measurable.

    \item If $\phi \in \mathcal{C}(\R)$ and $f$ is measurable, then $\phi \circ f$ is measurable.
    \begin{proof}
    
    \end{proof}

    \begin{remark}
    
    \end{remark}

    \item If $f,g$ are measurable, then $f \cdot g$ is measurable.
    \begin{proof}
    
    \end{proof}

    \item If $f,g$ are measurable and $g(x)  0 \forall x \in \Omega$, then $f \div g$ is measurable.
    \begin{proof}
      $f \div g = f \cdot \frac{1}{g}$ with $\frac{1}{g}$ being measurable by taking $\phi (x) = \frac{1}{x}$ in \textit{4}.
    \end{proof}
  \end{enumerate}
\end{theorem}

\begin{remark}
  \underline{Conclusion}: Arithmetric operations with measurable functions give measurable functions.
\end{remark}

\clearpage



\subsubsection{Almost Everywhere Properties}

\begin{definition}
  (Almost Everywhere)

  Let $(\Omega, \mathcal{A}, \mu)$ be a measure space with a complete measure.
  
  Then we say that a property of some points $\{x\ \in \Omega\}$ holds \textbf{almost everywhere (a.e.)} if and only if the property holds that $\forall x \in \Omega \setminus E$, where $\mu(E) = 0$.

  We say that a property of some points $\{x\ \in \Omega\}$ holds \textbf{almost everywhere (a.e.) on A}, where $A \in \mathcal{M}(\Omega)$, if and only if the property holds for $\forall x \in A \setminus E$, where $\mu(E) = 0$.
\end{definition}

\begin{example}
  \begin{enumerate}
    \item Dirichlet function: $D(x) = \begin{cases} 1, & x \in \Q, \\ 0, & x \in [0,1] \setminus \Q. \end{cases}$

    One can esaily check that 
    \item One can consider \textbf{convergence a.e.}: $f_n(x) \rightarrow f(x)$ a.e.
    
    \item One can consider funtions defined a.e.:
    
    \item Finally,instead of actual functions, we may consider their equivalent classes:
    
  \end{enumerate}
\end{example}

\begin{lemma}
  If $f$ is measurable and $\mu(A) = 0$,
  
  then if we define:
  $g(x) = \begin{cases} f(x), & x \in A, \\ 0,  & x \notin A.\end{cases}$, $g$ is still measurable.
  \begin{proof}
  
  \end{proof}
\end{lemma}

\begin{theorem}
  Let $\{f_n\}_{n=1}^{\infty}$ be a sequence of measurable fucntions on $(\Omega, \mathcal{A}, \mu)$ and $f_n(x) \overset{a.e.}{\rightarrow} f(x)$. Then $f(x)$ is also measurable.
  \begin{proof}

  \end{proof}
\end{theorem}

\begin{corollary}
  Let $\{f_n(x)\}$ be a sequence of measurable functions. If $f_n(x)$ is bounded from above $\forall n$ for a.e. $x \in \Omega$, then

  \begin{enumerate}
    \item $\sup\limits_{n} f_n(x)$\ is measurable;
    \item $\limsup\limits_{n \to \infty} f_n(x)$ is measurable;
  \end{enumerate}

  If, otherwise, $f_n(x)$ is bounded from below $\forall n$ for a.e. $x \in \Omega$, then
  \begin{enumerate}
    \item $\inf\limits_{n} f_n(x)$ is measurable;
    \item $\liminf\limits_{n \to \infty} f_n(x)$ is measurable.
  \end{enumerate}

  \begin{proof}
    Consider $g_1(x) := f_1(x)$, $g_2(x) := \max\{f_1(x), f_2(x)\} = \frac{|f_1(x) - f_2(x)| + f_1(x) + f_2(x)}{2}$, $g_3(x) := \max\{f_1(x), f_2(x), f_3(x)\} = \max\{g_2(x), f_3(x)\}$, \ldots, which are all measurable by properties in the last subsubsection.

    Then, $\sup\limits_{n} f_n(x) \overset{a.e.}{=} \lim\limits_{n \to \infty} g_n(x)$ is measurable.

    \underline{Recall}: $\limsup\limits_{n \to \infty} a_n(x) = \sup \{$limits of convergent subsequences$\} = \lim\limits_{k \to \infty} (\sup\limits_{n \geq k} a_n(x))$.

    Then, $\limsup\limits_{n \to \infty} f_n(x) = \lim\limits_{k \to \infty} (\sup\limits_{n \geq k} f_n(x))$ is measurable.

    Analogously, $\inf\limits_{n} f_n(x) = - \sup\limits_{n} (-f_n(x))$ is measurable; $\liminf\limits_{n \to \infty} f_n(x) = - \limsup\limits_{n \to \infty} (-f_n(x))$ is measurable.
  \end{proof}
\end{corollary}

\clearpage

\subsubsection{Egorov's Theorem}

\begin{theorem}
  (Egorov's Theorem)

  Let $(\Omega, \mathcal{A}, \mu)$ be a space with a \underline{finite} complete measure, and $\{f_n\}$ is a sequence of measurable functions with $f_n \overset{a.e.}{\rightarrow} f$.
  
  Then, $\forall \delta > 0$, $\exists$ a set $E_\delta \subset \Omega$ s.t. $\mu(E_\delta) < \delta$ and $f_n \overset{\Omega \setminus E_\delta}{\rightrightarrows} f$.

  \begin{proof}
    Fix $\delta > 0$. Consider the divergence set $E$ ($\mu(E) = 0$ by our assumption):

    $E = \bigcup_{k \geq 1} $
  \end{proof}
\end{theorem}

\begin{remark}
    Intuition here: on a set with small measure, convergence may be bad; but on the rest part with large measure, convergence is uniform.
\end{remark}

\begin{remark}
  Ths Egorov's Theorem may fail if $\mu(\Omega) = \infty$.

  \underline{Counter example 1}: Take $\Omega = \R$ with Lebesgue measure, $f_n(x) = \frac{x}{n}$.

  Then, $f_n(x) \overset{a.e.}{\rightarrow} 0$ on the whole real line, but $\forall E_\delta$ with finite measure, $f_n \not \rightrightarrows 0$ on $\R \setminus E_\delta$.

  \underline{Counter example 2}: Take $\Omega = \R$ with Lebesgue measure, $f_n(x) = \chi_{[n, n+1]}(x)$.

  Then, $f_n(x) \overset{a.e.}{\rightarrow} 0$ on the whole real line, but $\forall E_\delta$ with finite measure, $f_n \not \rightrightarrows 0$ on $\R \setminus E_\delta$.
\end{remark}

\begin{remark}
  In Egorov's Theorem, one CANNOT take $E_\delta = 0$.

  \underline{Counter example}: Take $\Omega = [0,1]$ with Lebesgue measure, $f_n(x) = x^n$.

  Then, $f_n(x) \overset{a.e.}{\rightarrow} 0$ for $x \in [0,1)$ and $f(1) = 1$.
\end{remark}

\begin{proposition}
  Let $E \subset \R$ be a closed set, $f \in \mathcal{C}(E)$. Then $\exists g \in \mathcal{C}(\R)$, s.t. $g|_E = f|_E$.
  \begin{proof}
    Since $E$ is closed, $\R \setminus E$ is open. So, we can write $\R \setminus E = \bigsqcup_{k=1}^{\infty} I_k$, where $I_k = (a_k, b_k)$.

    On each $I_k$, we define $g$ as the linear function connecting $(a_k, f(a_k))$ and $(b_k, f(b_k))$. Explicitly, $g(x) = f(a_k) + \frac{f(b_k) - f(a_k)}{b_k - a_k}(x - a_k), x \in I_k$.

    If there are some intervals which contain $\infty$ or $-\infty$, we just extend $g$ as a constant function on them.
    
    So, such defined $g$ is continuous on $\R$ and $g|_E = f|_E$.
  \end{proof}
\end{proposition}

\begin{remark}
  What's good about such linear link / extension?

  Linear functions not only preserves continuity, but also linear control, which may provide us with some sort of convenience in some problems.
\end{remark}

\begin{remark}
  This also works for $E \subset \R^n$, which requires a more complicated proof.
\end{remark}




\clearpage

\subsubsection{Lusin's Theorem}

\begin{theorem}
  (Lusin's Theorem)

  Let $f$ be a Lebesgue measurable function on $[a,b]$.

  Then $\forall \delta > 0$, $\exists E_\delta$ s.t. $\mu(E_\delta) < \delta$ and $\exists$ a continuous function $g \in \mathcal{C}([a,b])$, s.t. $f|_{[a,b] \setminus E} = g|_{[a,b] \setminus E}$.

  \begin{proof}
  
  \end{proof}
\end{theorem}

\begin{remark}
  $[a,b]$ can be replaced by any interval $I \subset \R$.
\end{remark}

\begin{remark}
  We could say instead of $f|_{[a,b] \setminus E} = g|_{[a,b] \setminus E}$ that $f$ is continuous on $[a,b] \setminus E$ for an open set $E$ (as follows from the proof).
\end{remark}

\begin{remark}
  The Lusin's Theorem also holds in $\R^n$ analogously: for $f$ - measurable on an open $G \subset \R^n$.
\end{remark}

\begin{remark}
  We still CANNOT take $E$ with $\mu(E) = 0$.

  \underline{Counter Example}: $f(x) = \begin{cases} \frac{1}{x}, & x \in (0,1]; \\ 0, & x = 0 \end{cases}$
\end{remark}

\begin{theorem}
  (Inverse Lusin's Theorem)

  Let $f$ be a function on $[a,b]$ with the \textbf{Lusin property} ($\forall \delta > 0$, $\exists E_\delta : \mu(E_\delta) < \delta$ and $g_\delta \in \mathcal{C}([a,b]): f|_{[a,b] \setminus E} = g|_{[a,b] \setminus E}$).

  Then $f$ is Lebesgue measurable.

  So, $f$ on $[a,b]$ is Lebesgue measurable $\Leftrightarrow$ $f$ has the Lusin property.
  \begin{proof}



  \end{proof}
\end{theorem}

\begin{remark}
  We established, in particular, that a measurable function on an interval is an a.e. limit of continuous functions.
\end{remark}

\clearpage


\subsubsection{Convergence in Measure}

\begin{definition}
  (Convergence in Measure)

  Let $f_n$ be a sequence of measurable functions on a measure space $(\Omega, \mathcal{A}, \mu)$ with a complete measure. We say that $f_n$ converges to $f$ \textbf{in measure} if $\forall \delta > 0$,
  $$\lim_{n \to \infty} \mu(\{x \in \Omega: |f_n(x) - f(x)| \geq \delta\}) = 0.$$

  In the theory of probability, this is also called \textbf{convergence in probability}. Notation: $X_n \xrightarrow{p} X$ where $X_n, X$ are all random variables.
\end{definition}





\clearpage

\section{Lebesgue Integration}




\clearpage




\appendix

\titleformat{\section}{\large\bfseries}{Appendix \thesection}{0.75em}{}
\titlespacing*{\section}{0pt}{1ex plus .5ex}{0.5ex}

\section{Biographies of Mathematicians}

For more biographies of mathematicians, you can go to \href{https://mathshistory.st-andrews.ac.uk/}{MacTutor}, which is a free online resource containing biographies of more than 3000 mathematicians and over 2000 essays and other supporting materials. The website is maintained by \href{https://www.st-andrews.ac.uk/mathematics-statistics/}{School of Mathematics and Statistics}, \href{https://www.st-andrews.ac.uk/}{University of St Andrews, Scotland}.

Copyright information: Almost all contents on \href{https://mathshistory.st-andrews.ac.uk/}{MacTutor} are lincensed under the \href{https://creativecommons.org/licenses/by-sa/4.0/}{Creative Commons Attribution-ShareAlike 4.0 International License}. Thank you \href{https://mathshistory.st-andrews.ac.uk/}{MacTutor}!

\clearpage

\subsection{Henri Lebesgue (1875 - 1941)}

\textbf{Henri Lebesgue}'s father was a printer. Henri began his studies at the Collège de Beauvais, then he went to Paris where he studied first at the Lycée Saint Louis and then at the Lycée Louis-le-Grand.

Lebesgue entered the École Normale Supérieure in Paris in 1894 and was awarded his teaching diploma in mathematics in 1897. For the next two years he studied in its library where he read Baire's papers on discontinuous functions and realised that much more could be achieved in this area. Later there would be considerable rivalry between Baire and Lebesgue which we refer to below. He was appointed professor at the Lycée Centrale at Nancy where he taught from 1899 to 1902. Building on the work of others, including that of Émile Borel and Camille Jordan, Lebesgue formulated the theory of measure in 1901 and in his famous paper \textit{Sur une généralisation de l'intégrale définie (On a generalization of the definite integral)}, which appeared in the \textit{Comptes Rendus} on 29 April 1901, he gave the definition of the Lebesgue integral that generalises the notion of the Riemann integral by extending the concept of the area below a curve to include many discontinuous functions. This generalisation of the Riemann integral revolutionised the integral calculus. Up to the end of the 19th century, mathematical analysis was limited to continuous functions, based largely on the Riemann method of integration.


His contribution is one of the achievements of modern analysis which greatly expands the scope of Fourier analysis. This outstanding piece of work appears in Lebesgue's doctoral dissertation, \textit{Intégrale, longueur, aire (Integral, length, area)}, presented to the Faculty of Science in Paris in 1902, and the 130 page work was published in Milan in the Annali di Matematica in the same year. Having graduated with his doctorate, Lebesgue obtained his first university appointment when in 1902 he became mâitre de conférences in mathematics at the Faculty of Science in Rennes. This was in keeping with the standard French tradition of a young academic first having appointments in the provinces, then later gaining recognition in being appointed to a more junior post in Paris. On 3 December 1903 he married Louise-Marguerite Vallet and they had two children. However the marriage only lasted until 1916 when they were divorced.

One honour which Lebesgue received at an early stage in his career was an invitation to give the Cours Peccot at the Collège de France. He did so in 1903 and then received an invitation to present the Cours Peccot two years later in 1905. Lebesgue first fell out with Baire in 1904, when Baire gave the Cours Peccot at the Collège de France, over who had the most right to teach such a course. Their rivalry turned into a more serious argument later in their lives. Lebesgue wrote two monographs \textit{Leçons sur l'intégration et la recherche des fonctions primitives (Lectures on integration and research on primitive functions)} (1904) and \textit{Leçons sur les séries trigonométriques (Lectures on trigonometric series)} (1906) which arose from these two lecture courses and served to make his important ideas more widely known. However, his work received a hostile reception from classical analysts, especially in France. In 1906 he was appointed to the Faculty of Science in Poitiers and in the following year he was named professor of mechanics there.

Let us attempt to indicate the way that the Lebesgue integral enabled many of the problems associated with integration to be solved. Fourier had assumed that for bounded functions term by term integration of an infinite series representing the function was possible. From this he was able to prove that if a function was representable by a trigonometric series then this series is necessarily its Fourier series. There is a problem here, namely that a function which is not Riemann integrable may be represented as a uniformly bounded series of Riemann integrable functions. This shows that Fourier's assumption for bounded functions does not hold.

In 1905 Lebesgue gave a deep discussion of the various conditions Lipschitz and Jordan had used in order to ensure that a function $f(x)$ is the sum of its Fourier series. What Lebesgue was able to show was that term by term integration of a uniformly bounded series of Lebesgue integrable functions was always valid. This now meant that Fourier's proof that if a function was representable by a trigonometric series then this series is necessarily its Fourier series became valid, since it could now be founded on a correct result regarding term by term integration of series. As Hawkins writes:-

\enquote{\textit{In Lebesgue's work ... the generalised definition of the integral was simply the starting point of his contributions to integration theory. What made the new definition important was that Lebesgue was able to recognise in it an analytic tool capable of dealing with - and to a large extent overcoming - the numerous theoretical difficulties that had arisen in connection with Riemann's theory of integration. In fact, the problems posed by these difficulties motivated all of Lebesgue's major results.}}

He was appointed mâitre de conférences in mathematical analysis at the Sorbonne in 1910. During the first world war he worked for the defence of France, and at this time he fell out with Borel who was doing a similar task. Lebesgue held his post at the Sorbonne until 1918 when he was promoted to Professor of the Application of Geometry to Analysis. In 1921 he was named as Professor of Mathematics at the Collège de France, a position he held until his death in 1941. He also taught at the École Supérieure de Physique et de Chimie Industrielles de la Ville de Paris between 1927 and 1937 and at the École Normale Supérieure in Sèvres.

It is interesting that Lebesgue did not concentrate throughout his career on the field which he had himself started. This was because his work was a striking generalisation, yet Lebesgue himself was fearful of generalisations. He wrote:-

\enquote{\textit{Reduced to general theories, mathematics would be a beautiful form without content. It would quickly die.}}

Although future developments showed his fears to be groundless, they do allow us to understand the course his own work followed.

He also made major contributions in other areas of mathematics, including topology, potential theory, the Dirichlet problem, the calculus of variations, set theory, the theory of surface area and dimension theory. By 1922 when he published \textit{Notice sur les travaux scientifique de M Henri Lebesgue} he had written nearly 90 books and papers. This ninety-two page work also provides an analysis of the contents of Lebesgue's papers. After 1922 he remained active, but his contributions were directed towards pedagogical issues, historical work, and elementary geometry.

Lebesgue was honoured with election to many academies. He was elected to the Academy of Sciences on 29 May 1922, to the Royal Society, the Royal Academy of Science and Letters of Belgium (6 June 1931), the Academy of Bologna, the Accademia dei Lincei, the Royal Danish Academy of Sciences, the Romanian Academy of Sciences, and the Kraków Academy of Science and Letters. He was also awarded honorary doctorates from many universities. He also received a number of prizes including the Prix Houllevigue (1912), the Prix Poncelet (1914), the Prix Saintour (1917) and the Prix Petit d'Ormoy (1919).

\clearpage

\subsection{Constantin Carathéodory (1873 - 1950)}

\textbf{Constantin Carathéodory}'s father, Stephanos Carathéodory, was an Ottoman Greek who had studied law in Berlin and then served as secretary in the Ottoman embassies in Berlin, Stockholm and Vienna. Stephanos had married Despina Petrocochino, who came from a Greek family of businessmen who had settled in Marseille. At the time of Constantin's birth, the family were in Berlin since Stephanos had been appointed there two years earlier as First Secretary to the Ottoman Legation.

The Carathéodory family spent 1874-75 in Constantinople, where Constantin's paternal grandfather lived, while Stephanos was on leave. Then in 1875 they went to Brussels when Stephanos was appointed there as Ottoman Ambassador. In Brussels, Constantin's younger sister Loulia was born. The year 1895 was a tragic one for the family since Constantin's paternal grandfather died in that year, but much more tragically, Constantin's mother Despina died of pneumonia in Cannes. Constantin's maternal grandmother took on the task of bringing up Constantin and Loulia in his father's home in Belgium. They employed a German maid who taught the children to speak German. Constantin was already bilingual in French and Greek by this time.

Constantin began his formal schooling at a private school in Vanderstock in 1881. He left after two years and then spent time with his father on a visit to Berlin, and also spent the winters of 1883-84 and 1884-85 on the Italian Riviera. Back in Brussels in 1885 he attended a grammar school for a year where he first became interested in mathematics. In 1886 he entered the high school Athénée Royal d'Ixelles and studied there until his graduation in 1891. Twice during his time at this school Constantin won a prize as the best mathematics student in Belgium.

At this stage Carathéodory began training as a military engineer. He attended the École Militaire de Belgique from October 1891 to May 1895 and he also studied at the É d'Application from 1893 to 1896. In 1897 a war broke out between Turkey and Greece. This put Carathéodory in a difficult position since he sided with the Greeks, yet his father served the government of the Ottoman Empire. Since he was a trained engineer he was offered a job in the British colonial service. This job took him to Egypt where he worked on the construction of the Assiut dam until April 1900. During periods when construction work had to stop due to floods, he studied mathematics from some textbooks he had with him, such as Jordan's \textit{Cours d'Analyse (Course of analysis)} and Salmon's text on the analytic geometry of conic sections. He also visited the Cheops pyramid and made measurements which he wrote up and published in 1901. He also published a book Egypt in the same year which contained a wealth of information on the history and geography of the country.


Carathéodory entered the University of Berlin in May 1900 where Frobenius and Schwarz were professors. He attended Frobenius's lectures but benefited most from a twice monthly colloquium run by Schwarz who was lecturing on his collected works. He also became close friends with Fejér while at Berlin. After hearing excellent reports of mathematics research at Göttingen, he decided to continue his studies there and enrolled for the summer semester of 1902. Carathéodory was indeed impressed with Göttingen, describing it as the:-

\enquote{\textit{... seat of an international congress of mathematicians permanently in session.}}

He worked on the calculus of variations and was much influenced by both Hilbert and Klein. He received his doctorate in 1904 from Göttingen University for his thesis \textit{Über die diskontinuierlichen Lösungen in der Variationsrechnung (On discontinuous solutions in the calculus of variations)} which he submitted to Hermann Minkowski. His oral examination was held on 13 July when he was also examined in his subsidiary subjects of applied mathematics and astronomy by Klein and Schwarzschild. He remained at Göttingen to write his habilitation thesis \textit{Über die starken Maxima und Minima bei einfachen Integralen (On strong maxima and minima in simple integrals)} which he submitted on 5 March 1905. He then lectured as a Privatdozent at Göttingen until 1908.

Carathéodory had spent time in Brussels with his father Stephanos over the summer of 1907. After a few months of deteriorating health Stephanos died in late 1907. Study, at Bonn, had proposed Carathéodory as Furtwängler's successor and, after serious thought as to whether he should leave Göttingen, Carathéodory went to Bonn where he became a Provatdozent on 1 April 1908. At Bonn he collaborated with Study on isoperimetric problems. On 5 February 1909 he married Euphrosyne Carathéodory in Constantinople. In marrying Euphrosyne, who was his aunt and eleven years his junior, Carathéodory was following a family tradition of marrying close relatives.

After a year at Bonn, Carathéodory was appointed as Professor of higher Mathematics at the Technical University of Hanover, so becoming Stäckel's successor. Again it was not long before he moved on and on 1 October 1910 he was appointed to the Chair of Higher Mathematics at the Technical University of Breslau. This time he held the chair for two and a half years before being appointed professor at Göttingen from 1 April 1913. The years of World War I were difficult ones for Carathéodory and his family. Most of his colleagues and students served in the military and he was isolated in Göttingen. The famine of 1917 hit hard but Carathéodory continued to give lecture courses at the university.

After five years Göttingen he was appointed to the University of Berlin in 1918 but after he had been there for a year, at the request of the Greek government, he ended his contract with Berlin on 31 December 1919 and travelled to Greece to undertake a new venture. By this time Constantin and Euphrosyne had two children, Stephanos born in Hanover on 7 November 1909 and Despina born on 13 October 1912. Carathéodory had also accepted editorial positions on the boards of two major mathematics journals, the \textit{Rendiconti del Circolo Matematico di Palermo} from 1909 and the \textit{Mathematische Annalen} from 1914.

The Greek government had asked Carathéodory to establish a second university in Smyrna. However, he also required a university post so he was appointed as Professor of Analytical and Higher Geometry at the University of Athens on 2 June 1920. On 14 July the Greek government published a bill setting up a Greek University in Smyrna and soon others were appointed to assist Carathéodory. On 28 July Carathéodory was officially appointed as organiser of the Ionian University in Smyrna and also Professor of Mathematics at the new university. In the second half of 1921 he travelled widely through Europe purchasing books and equipment for the new university. The Turks attacked Smyrna in September 1922 and so the planned opening of the university in October of that year became impossible. Carathéodory was able to save the university library, which he had worked so hard to establish, and most of the equipment which he had purchased for the science departments, and escaped to Athens on a Greek battleship. He taught at Athens at the National University and the National Technical University until 1924 when he moved to Munich to fill the chair left vacant when Lindemann retired.

In 1928 Carathéodory became the first visiting lecturer of the American Mathematical Society. He sailed to the United States with his wife in January and after a lecture tour and time spent as a visiting professor at Harvard, returned to Munich in September. In the following year he received an offer of a post from Stanford university and was in fact appointed there in September 1929. However, he seems to have only been using this offer as a means of getting better salary and conditions from Munich, which indeed he managed to do.

On 30 January 1933 the National Socialist party led by Hitler came to power in Germany:-

\enquote{\textit{Carathéodory could hardly conceive how this could happen in a country with the cultural traditions of Germany. He initially tended to view the Hitler regime with a somewhat overconfident contempt, whereas later, when Hitler gained absolute power, he was incapable of resistance. His behaviour in the Nazi era was, in fact, identical with that of the ... educated bourgeois who, despite their humanistic background, in their overwhelming majority abstained from any opposition against Hitler's dictatorship, and especially Hitler's war, and thus dramatically failed to exercise their historic responsibility towards both Germany and humanity as a whole.}}

Carathéodory continued to hold his position in Munich until he retired in August 1938. However he certainly undertook many duties which took him to other places. In particular he continued to work on reorganising the Greek universities, particularly during 1930-32, with the aim of integrating Greece academically into Europe. In 1936-37 he made another visit to the United States, giving a lecture at the American Mathematical Society meeting to commemorate the tercentenary of Harvard University on 31 August 1936, then spending the winter semester at the University of Wisconsin as Carl Schurz Memorial Professor.

The World War II was a difficult time for Carathéodory. Georgiadou writes:-

\enquote{\textit{... during World War II he took part in the procedures of the Bavarian Academy of Sciences. He did not get involved in the movement for national socialism, but he did have connections with Nazi party members [particularly Hasse, Blaschke and Süss]. He never openly mentioned the holocaust or the Nazi crimes against Greece. ... kept silent in the face of crimes that violated any idea of human decency, accepted the authority of an illegal state, made his compromises and submitted to the expulsion of Jews from scientific institutions ... However, he took great pains to re-establish mathematics as an academic discipline in Germany after the war and thus to contribute to the reintegration of this country into the community of civilised nations.}}

Carathéodory made significant contributions to the calculus of variations, the theory of point set measure, and the theory of functions of a real variable. He added important results to the relationship between first order partial differential equations and the calculus of variations. He contributed important results to the theory of functions of several variables. He examined conformal representations of simply connected regions and he developed a theory of boundary correspondence. He also made contributions in thermodynamics, the special theory of relativity, mechanics, and geometrical optics.

Carathéodory wrote many fine books including \textit{Lectures on Real Functions} (1918), \textit{Conformal representation} (1932), \textit{Calculus of Variations and Partial Differential Equations} (1935), \textit{Geometric Optics} (1937), \textit{Real functions Vol. 1: Numbers, Point sets, Functions} (1939), and \textit{Funktionentheorie}, a 2 volume work published in 1950.

One might wonder why there is no \textit{Real functions Vol. 2} in this list. In fact Carathéodory did write the second volume of this work but it was destroyed while at the publisher Teubner during the bombing of Leipzig in 1943.

Perron, writing in 1952, remarks that Carathéodory:-

\enquote{\textit{... had not published many of his ideas; they result in others works, especially in those of the numerous students who were introduced by him to the spirit and ways of scientific research and who partly themselves occupy university chairs today.}}

He supervised two doctoral students at Göttingen (Hans Rademacher and Paul Finsler), one at Berlin, and 17 at Munich.


\end{document}

\documentclass[12pt]{article}

\usepackage{amsmath, amssymb, amsthm}
\usepackage{scalerel,stackengine}
\usepackage{esint}
\usepackage{geometry}
\geometry{a4paper, margin=1in}
\usepackage{setspace}
\onehalfspacing % 设置为1.5倍行距
\usepackage{hyperref}
\hypersetup{colorlinks=true, linkcolor=blue, citecolor=magenta, urlcolor=cyan}
\usepackage{titlesec}

\titleformat{\section}
{\normalfont\bfseries\LARGE}
{\thesection}{1em}{}
\titleformat{\subsection}
{\normalfont\bfseries\Large}
{\thesubsection}{1em}{}
\titleformat{\subsubsection}
{\normalfont\itshape\bfseries\large}
{\thesubsubsection}{1em}{}

\theoremstyle{plain}
\newtheorem{theorem}{Theorem}[section]
\newtheorem{lemma}[theorem]{Lemma}
\newtheorem{proposition}[theorem]{Proposition}
\newtheorem{corollary}[theorem]{Corollary}

\theoremstyle{definition}
\newtheorem{definition}[theorem]{Definition}
\newtheorem{example}[theorem]{Example}

\theoremstyle{remark}
\newtheorem{remark}[theorem]{Remark}

\newcommand{\R}{\mathbb{R}}
\newcommand{\N}{\mathbb{N}}
\newcommand{\C}{\mathbb{C}}
\newcommand{\Z}{\mathbb{Z}}
\newcommand{\Q}{\mathbb{Q}}
\newcommand{\eps}{\varepsilon}

\numberwithin{equation}{section}

\usepackage{fancyhdr}

\pagestyle{fancy}
\fancyhf{}
\fancyhead[C]{MA337 Real Analysis (H) Notes}
\fancyfoot[C]{\thepage}
\setlength{\headheight}{14pt}
\setlength{\headsep}{20pt}
\setlength{\footskip}{25pt}
\fancypagestyle{plain}{
  \fancyhf{}
  \fancyhead[C]{MA337 Real Analysis (H) Notes}
  \fancyfoot[C]{\thepage}
}

\begin{document}

\title{MA337 Real Analysis (H) Notes}
\author{Kai Chen}
\date{\today}

\maketitle

\begin{abstract}
  These notes are compatible to the MA337 course (2025 Fall).
\end{abstract}

\clearpage
\pagenumbering{roman}
\tableofcontents

\clearpage
\pagenumbering{arabic}

\section{Set Theory}

\clearpage

\section{Metric Spaces}

\clearpage

\section{Continuous Maps}

\clearpage

\section{Compactness}

\clearpage

\section{Lebesgue Measure and Integration}
We now enter the chaper of Lebesgue measure and integration.

In this chapter, we embark on a journey into the fundamental concepts of Lebesgue measure and integration. This theory, developed by Henri Lebesgue at the beginning of the 20th century, provides a more robust and general framework for integration than the Riemann integral, allowing us to integrate a wider class of functions and providing powerful convergence theorems essential for modern analysis and probability theory. We will begin by establishing the foundational building blocks: systems of sets, such as semi-rings, rings, algebras, and $\sigma$-algebras, which provide the structure upon which measures are defined.

We will then introduce the concept of a measure, focusing on $\sigma$-additivity, a crucial property that allows for the consistent measurement of infinite unions of disjoint sets. A significant part of this chapter will be dedicated to the construction of the Lebesgue measure on Euclidean spaces, starting from elementary "volumes" of semi-open cells and extending it to a much broader collection of sets through the process of outer measure and Carathéodory's extension theorem. This rigorous construction will lead us to the definition of measurable sets and measurable functions, paving the way for the development of the Lebesgue integral in subsequent chapters.

\clearpage

\subsection{Systems of Sets: Semi-rings, Rings, Algebras, $\sigma$-Algebras, Borel $\sigma$-Algebra, Measures}

\begin{definition}
  (Semi-Ring of Sets)

  A system of sets $S$ is called a \textbf{semi-ring} if it satisfies the following two axioms:
  \begin{enumerate}
    \item If $A, B \in S$, then $A \cap B \in S$.
    \item If $A, B \in S$, then there exist disjoint sets $A_1, A_2, \ldots, A_n \in S$ such that 
    
    $A \setminus B = \bigsqcup_{i=1}^n A_i$.
  \end{enumerate}
\end{definition}

\begin{example}
  (semi-open cells in $\R^n$)

  $I_1, I_2, \ldots, I_n$: intervals in $\R$.
  $C:=I_1 \times I_2 \times \ldots \times I_n$ is called a \textbf{cell} in $\R^n$.

  \textbf{semi-open interval}: an interval that is closed at one end and open at the other end, e.g., $[a, b)$ or $(a, b]$.

  Let $S$ be the collection of all semi-open cells in $\R^d$ (not required to be finite!), i.e. $S = \{[a_1,b_1)\times\cdots\times[a_n,b_n): a_i,b_i\in\mathbb{R}, a_i<b_i\}$. Then $S$ is a semi-ring.

  \underline{Warning}: Be cautious about the directions of semi-open cells! The directions of all cells must coincide.
\end{example}

\begin{remark}
  \underline{Question}: Can we take all closed/open cells in $R^n$?

  \underline{Answer}: NO! For example, $[0,1] \cap [1,2] = \{1\}$, $(0,1) \setminus (1/2,1) = (0,1/2]$, both result in some elements not in the original system.

\end{remark}

\begin{proposition}
  If $S$ is a semi-ring, then
  \begin{enumerate}
    \item $\emptyset \in S$.
    \item Axiom 2 can be strenghtened to: $\forall A \in S$, $\forall A_1, A_2, \ldots, A_n \in S, A_j \in A, \forall j, disjoint$, there exist disjoint sets $A_{m+1}, A_{m+2}, \ldots, A_s \in S$ such that $A = \bigsqcup_{i=1}^s A_i$.
  \end{enumerate}
  \begin{proof}
    1. $\emptyset = A \setminus A. \forall A \in S$.

    2. One can prove by induction on $m$: splitting the whole area $A$ into disjoint parts. It is easier to prove for the semi-ring $\{$all cells in $\R^n\}$.
  \end{proof}
\end{proposition}

\begin{remark}
  We now show that with axiom 1 and the strenghtened condition above we could say $S$ is a semi-ring.
  \begin{proof}
    Now axiom 1 is satisfied.

    Suppose $A, B \in S$, then $A \setminus B = A \ (A \cap B)$. Let $A_1 = B, n = 1$. By our strenghthened condition, one could find disjoint sets $A_2, A_3, \ldots, A_s \in S$, s.t. $A = \bigsqcup_{i=1}^s A_i$, i.e. $A \setminus B = \bigsqcup_{i=2}^s A_i$. $\checkmark$
  \end{proof}
  Thus, we have the following equivalent definition for semi-rings.
\end{remark}

\begin{definition}
  (Semi-Ring of Sets - Alternative Definition)

  A system of sets $S$ is called a \textbf{semi-ring} if it satisfies the following two axioms:
  \begin{enumerate}
    \item If $A, B \in S$, then $A \cap B \in S$.
    \item $\forall A \in S$, $\forall A_1, A_2, \ldots, A_n \in S, A_j \subset A, \forall j, disjoint$, there exist disjoint sets

      $A_{m+1}, A_{m+2}, \ldots, A_s \in S$ such that $A = \bigsqcup_{i=1}^s A_i$.
  \end{enumerate}
\end{definition}

\begin{definition}
  (Semi-ring with Unity)

  A semi-ring $S$ is called a \textbf{semi-ring with unity} if $S \in 2^\Omega (\leftrightarrow \forall A \in S, A \in \Omega) $ and $\Omega \in S$ for some set $\Omega$. $\Omega$ is called the \textbf{unity} of $S$. Indeed, $\Omega \cap A = A, \forall A \in S$.
\end{definition}

\begin{example}
    \begin{enumerate}
        \item (a semi-ring with unity)
        
        The semi-ring of all semi-open cells in $\R^n$ (to be more precise, we need to add the element $\R^n$ into it) is a semi-ring with unity $\R^n$.
        \item (a semi-ring WITHOUT a unity)
        
        The semi-ring of all finite semi-open cells in $\R^n$: NO unity ($\R^n$)!
    \end{enumerate}
\end{example}

\begin{definition}
  (Ring of Sets)

  A system of sets $\mathcal{R}$ is called a \textbf{ring} if it satisfies the following two axioms:
  \begin{enumerate}
    \item $\forall A, B \in \mathcal{R}$, $A \cap B \in \mathcal{R}$.
    \item $\forall A, B \in \mathcal{R}$, $A \triangle B = (A \setminus B) \cup (B \setminus A) \in \mathcal{R}$.
  \end{enumerate}
\end{definition}

\begin{remark}
  In fact, a ring $R$ is closed under set difference and finite unions.
  \begin{enumerate}
    \item $\forall A, B \in R$, $A \setminus B = A \triangle (A \cap B) \in R$.
    \item $\forall A, B \in R$, $A \cup B = (A \triangle B) \triangle (A \cap B) \in R$.
  \end{enumerate}

  Conversely, we have
  \begin{enumerate}
    \item $\forall A, B \in R$, $A \cap B =  ((A \cup B) \setminus (A \setminus B)) \setminus (B \setminus A)\in R$.
    \item $\forall A, B \in R$, $A \triangle B = (A \cup B) \setminus (A \cap B) \in R$.
  \end{enumerate}
\end{remark}

\begin{definition}
  (Ring of Sets - Alternative Definition)

  A system of sets $\mathcal{R}$ is called a \textbf{ring} if it satisfies the following two axioms:
  \begin{enumerate}
    \item $\forall A, B \in \mathcal{R}$, $A \setminus B \in \mathcal{R}$.
    \item $\forall A, B \in \mathcal{R}$, $A \cup B \in \mathcal{R}$.
  \end{enumerate}
\end{definition}

\begin{remark}
  As a result, we arrive with the same definition of ring requiring closeness under set difference and finite unions.
\end{remark}

\begin{example}
  (a semi-ring but NOT a ring)

  The semi-ring of all cells in $\R^n$: not ensuring the closeness under union!
\end{example}

\begin{definition}
    (Algebra)
    
    A ring with unity is called an \textbf{algebra of sets}.
\end{definition}

\begin{example}
  (a ring but NOT an algebra)

  Consider $R = \{ A \subset \N : |A| < +\infty \}$.
  $R$ is a ring, but $\N \notin R$, which means it doesn't have a unity.
\end{example}

\begin{proposition}
  \begin{enumerate}
    \item A ring is a semi-ring.
    \item $\forall$ system of sets $P$, $\exists$ a \textbf{minimal ring} $\mathcal{R}(P) \supset P$.
  \end{enumerate}
  \begin{proof}
    1. Let $\mathcal{R}$ be a ring. Then $\forall A, B \in \mathcal{R}$, $A \setminus B = A \setminus B (!)= A \triangle (A \cap B) \in \mathcal{R}.$

    2. Start with $\mathcal{R}_0 = 2^\Omega$, where $\Omega$ is the union of all sets in $P$. Let $\{R_\alpha\}$ be the collection of all rings containing $P$. Then
    \( \mathcal{R}(P) := \bigcap_\alpha R_\alpha \) is the minimal ring containing $P$ (it is clearly again a ring!).
  \end{proof}
\end{proposition}

\begin{proposition}
  Let $S$ be a semi-ring, then
  $$\mathcal{R}(S) = \{\bigcup_{j=1}^m A_j, A_j \in S, m \in \N: arbitrary\} \Leftrightarrow \{\bigsqcup_{j=1}^s A_j, A_j \in S, s \in \N: arbitrary\}$$
  \begin{proof}
    "$\Leftrightarrow$":

    Firstly, the claimed system $\mathcal{R}(S)$ is indeed a ring.

    $A = \sqcup_{j=1}^S A_j, B = \sqcup_{i = 1}^m B_i, A \cap B = \sqcup_{i,j}(A_j \cap B_i) \in S \subset \mathcal{R}(S)$.

    $\Rightarrow A \triangle B = (A \setminus B) \cap (B \setminus A)  = \sqcup_{j=1}^m \cap_{i=1}^m (A_j \setminus B_i) \in S \subset \mathcal{R}(S)$.

    Thus, $\mathcal{R}(S)$ is a ring.

    Next, $\forall$ other ring $\tilde{\mathcal{R}}(S)$ containing $S$, it must contain all elements of $\mathcal{R}(S)$.

    i.e. $\tilde{\mathcal{R}}(S) \supset \mathcal{R}(S)$ $\Rightarrow \mathcal{R}(S)$ is the minimal ring containing $S$.
  \end{proof}
\end{proposition}

\begin{definition}
  ($\sigma$-algebra)

  A system of sets $\mathcal{A}$ is called a \textbf{$\sigma$-algebra} if $\mathcal{A} \subset 2^{\Omega}, \Omega \in \mathcal{A}, \mathcal{A}$ is an algebra with unity $\Omega$, and $\forall A_1, A_2, \ldots$ (finite or infinite family of sets!) with $\forall j: A_j \in \mathcal{A}$ it holds $\cup_{j=1}^{\infty} A_j \in \mathcal{A}$.
\end{definition}

\begin{proposition}
  \begin{enumerate}
    \item A $\sigma$-algebra is closed under taking the implement: $A^c = \Omega \setminus A \in \mathcal{A}$ since a $\sigma$-algebra is a ring with unity $\Omega$. It is closed under set difference.
    \item $\emptyset \in \mathcal{A}$ since $\emptyset = \Omega^c$ or $\emptyset = \Omega \setminus \Omega$.
    \item A $\sigma$-algebra is closed under finite or countable union thanks to its definition and the fact that $\emptyset \in \mathcal{A}$
    \item A $\sigma$-algebra is closed under finite or countable intersection:

      $\forall A_1, A_2, \ldots$ (finite or infinite family of sets!) with $\forall j: A_j \in \mathcal{A}$, we have

      $\cap_{j=1}^{\infty} A_j = \Omega \setminus \cup_{j=1}^{\infty} (\Omega \setminus A_j) \in \mathcal{A}$
    \item A $\sigma$-algebra is closed under countable symmetric difference.
  \end{enumerate}
\end{proposition}

\begin{remark}
  \underline{Question}: We have so many seemingly equilvalent conditions for the definition of a $\sigma$-algebra, what are the least number of conditions we need to define/prove a $\sigma$-algebra?

  \underline{Answer}: I prefer the following three minimal conditions:

  1. $\Omega \in \mathcal{A}$.

  2. If $A \in \mathcal{A}$, then $A^c = (\Omega \setminus A) \in \mathcal{A}$.

  3. If $A_1, A_2, \ldots \in \mathcal{A}$, then $\bigcup_{j=1}^\infty A_j \in \mathcal{A}$.
\end{remark}

\begin{proposition}
  $\forall S \in 2^\Omega, \exists !$ minimum $\sigma$-algebra $\mathcal{A}(S) \supset S$.
  \begin{proof}
    Similar as the proof for $\mathcal{R}(S)$.
  \end{proof}
\end{proposition}

\begin{remark}
  \underline{Upshot 1}:

  In general:

  System of sets$\Rightarrow$ Semi-ring $\Rightarrow$ Ring $\Rightarrow$ Algebra with unity $\Rightarrow \sigma$-Algebra.

  Now, start with a semi-ring with unity $S$

  $\rightarrow$ could generate a ring $\mathcal{R}(S)$ (still equipped with a unity $\Omega$)

  $\rightarrow$ A ring with unity is actually an algebra with unity!

  $\rightarrow$ An algebra of sets: $\mathcal{A}(R(S)) = \mathcal{A}(S)$.
\end{remark}

\begin{remark}
  \underline{Upshot 2}:

  A system of sets $S$

  $\rightarrow$ ensuring the two axioms: closeness under intersection and being able to be decomposed into some disjoint subsets

  $\rightarrow$ A semi-ring!

  $\rightarrow$ could generate a ring $\mathcal{R}(S)$!

  $\rightarrow$ A ring which satisfies closeness under: (intersection and symmetric difference) or (union and difference)

  $\rightarrow$ equip with a unity

  $\rightarrow$ An algebra of sets!
\end{remark}

\begin{definition}
  (Borel $\sigma$-algebra)

  The \textbf{Borel $\sigma$-algebra} on $\R^n$ is defined as the minimum $\sigma$-algebra containing all open sets in $\R^n$, denoted as $\mathcal{B}(\R^n)$.
\end{definition}

\begin{remark}
  Note that $\mathcal{B}(\R^d)$ also contains all closed sets in $\R^d$ since it is closed under difference (open $\rightarrow$ semi-open $\rightarrow$ closed).

  Thus, an alternate definition of $\mathcal{B}(\R^d)$ is the minimum $\sigma$-algebra containing all closed sets in $\R^d$.
\end{remark}

\begin{definition}
  (Measure on a Semi-Ring)

  Let $S$ be a semi-ring. A function $\mu: S \rightarrow [0, + \infty)$ is called a \textbf{(finitely additive) measure on $S$} if it satisfies the following two axioms:
  \begin{enumerate}
    \item (Non-negativity) $\forall A \in S, \mu(A) \geq 0$.
    \item (Finite Additivity) If $A, A_1, A_2, \ldots, A_n \in S$ such that $A = \bigsqcup_{j=1}^n A_j$, then
      \(\mu(A) = \sum_{j=1}^n \mu(A_j).\)
  \end{enumerate}
\end{definition}

\begin{proposition}
  \begin{enumerate}
    \item $\mu(\emptyset) = 0$.
    \item $\forall A, B \in S, A \subset B$, we have $\mu(A) \leq \mu(B)$.
  \end{enumerate}
  \begin{proof}
    \begin{enumerate}
      \item $\emptyset = \emptyset \cup \emptyset \Rightarrow \mu(\emptyset) = 2\mu(\emptyset)$.
      \item Since $S$ is a semi-ring, there exist $A_1, A_2, \ldots, A_m \in S$, s.t. $B \setminus A = \bigsqcup_{l=1}^p A_j$

        $\Rightarrow B = A \bigsqcup (\bigsqcup_{j=1}^p A_j) \Rightarrow \mu(B) = \mu(A) + \Sigma_{j=1}^p \mu(A_j) \geq \mu(A)$.
    \end{enumerate}
  \end{proof}
\end{proposition}

\begin{example}
  On the semi-ring $\{$all finite semi-open cells in $\R^n\}$, we define a measure as follows:

  A finite semi-open cell $C = I_1 \times I_2 \times \ldots \times I_n$ in $\R^n$, define $\mu(C) := l(I_1) \times l(I_2) \times \ldots \times l(I_n)$, where $l(I) := $length of $I$ and we are measuring the cell's "volume".

  This $\mu$ is called the \textbf{Lebesgue measure on all finite semi-open cells in $\R^n$}.
\end{example}

\begin{proposition}
  $\forall$ measure on a semi-ring $S$ can be extended (with identical proerties) to $R(S)$.
  \begin{proof}
    For $A = \sqcup_{j=1}^m A_j \in \mathcal{R}(S)$ with $A_j \in \mathcal{R}(S)$, define $\mu(A) := \Sigma_{j=1}^m \mu(A_j)$. (We need to firstly deal with $A_j \in S$, and then gradually scan the whole $\mathcal{R}(S)$ based on measure-already-defined sets.)

    \underline{Well-defined (Correctness)}: Suppose $A = \sqcup_{j=1}^p A_j = \sqcup_{i=1}^s A_i^{\prime}$. We have

    $\Sigma_{j=1}^p \mu(A_j) = \{$using the finite additivity of $\mu$, and $A_j = A_j \cap A = \sqcup_{i=1}^s (A_j \cap A_i^{\prime})\}$ $= \Sigma_{j=1}^p (\Sigma_{i=1}^s \mu(A_j \cap A_i^{\prime})) = \Sigma_{i=1}^s (\Sigma_{j=1}^p \mu(A_i^{\prime} \cap A_j)) = \Sigma_{i=1}^s \mu(A_i^{\prime})$. $\checkmark$

    \underline{Non-negativity}: Clearly, $\mu(A) \geq 0$. $\checkmark$

    \underline{Finite Additivity}: Suppose $A, B \in R(S): A \cap B = \emptyset$.
    $A = \sqcup_{j=1}^p A_j, B = \sqcup_{i=1}^q B_j$, with $A_j, B_i \in S$.

    $\Rightarrow A \sqcup B = (\sqcup_{j=1}^p A_j) \sqcup (\sqcup_{i=1}^q B_i)$

    $\Rightarrow \mu(A \sqcup B) = \Sigma_{j=1}^p \mu(A_j) + \Sigma_{i=1}^q \mu(B_i)$

    Same for finite union of sets. $\checkmark$
  \end{proof}
\end{proposition}

\begin{proposition}
  (Proerties of a Measure on a ring $\mathcal{R}$)

  \begin{enumerate}
    \item $\mu(\emptyset) = 0$.
    \item If $A, B \in R, A \subset B$, then $\mu(A) \leq \mu(B)$.
    \item (\textbf{Semi-Additivity}) If $A \subset \cup_{j=1}^n A_j$, with $A, A_j \in R$, then $\mu(A) \leq \Sigma_{j=1}^n \mu(A_j)$.

      Now, switch from $\bigcup_{j=1}^n$ to $\bigsqcup_{j=1}^n$:

      Set $A_1^{\prime} := A_1, A_2^{\prime} := A_2 \setminus A_1, A_3^{\prime} := A_3 \setminus \cup_{j=1}^2 A_j, \ldots$

      Now, we have $\bigcup_{j=1}^n A_j = \bigsqcup_{j=1}^n A_j^{\prime}$.

      Thus, $A \subset \bigsqcup_{j=1}^n A_j^{\prime}$ (even more: $A = (\bigsqcup_{j=1}^n A_j^{\prime}) \bigcap A = \bigsqcup_{j=1}^n (A_j^{\prime} \bigcap A)$ !).

      Then, \(\mu(A) = \bigsqcup_{j=1}^n \mu(A_j^{\prime} \bigcap A) \leq \Sigma_{j=1}^n \mu(A_j^{\prime}) \leq \Sigma_{j=1}^n \mu(A_j).\)
  \end{enumerate}
\end{proposition}

\begin{remark}
  \underline{Question}: Could prop. 5.30 (3) maintain for a measure on a semi-ring? Why?

  \underline{Answer}: NO!!! The key difference between a semi-ring and a ring is that: in a semi-ring $S$, the diffence between sets may not belong to $S$, which means though they could be represented as disjoint unions of sets in $S$, they do NOT have measure defined on them! Then the inequality chain cannot go forward anymore.
\end{remark}

\clearpage

\subsection{Lebesgue Extension of a $\sigma$-Additive Measure}

\begin{definition}
    ($\sigma$-additivity)
    
    A measure $\mu$ on a semi-ring $S$ is called to satisfy \textbf{$\sigma$-additivity} 
    
    \textbf{(countable-additivity)} if for any $A \in S$, $\{A_j\}_{j=1}^{\infty} \subset S$ such that $A = \bigsqcup_{j=1}^\infty A_j$, we have
  \(\mu(A) = \sum_{j=1}^\infty \mu(A_j).\)
\end{definition}

\begin{remark}
    A $\sigma$-algebra is not necessarily $\sigma$-additive! 
    
    Also note that $\sigma$-additivity always implies
    
    \textbf{semi-$\sigma$-additivity}: $\forall A \subset \bigcup_{j=1}^{\infty} A_j, A,A_j \in S, \mu(A) \leq \Sigma_{j=1}^{\infty}\mu(A_j)$.

    And more importantly, finite additivity implies semi-$\sigma$-additivity also!
\end{remark}

\begin{example}
  \begin{enumerate}
    \item Let $\Omega = \N$, $S = 2^\Omega$. Define $\mu(A) := \Sigma_{j \in A} p_j$, where $p_j$ is the ''weight'' assigned to element $j \in \N$ satisfying $\Sigma_{j=1}^\infty p_j = 1$ (or any finite number). Then $\mu$ is a $\sigma$-additive measure on $S$.
    \item Let $\Omega = \N$, $S = 2^\Omega$. Define $\mu(A) := |A|$ (if $A$ is infinite, $\mu(A) := +\infty$). Then $\mu$ is a $\sigma$-additive measure on $S$. (View "weight" being $1$ for all elements. This is the case violating the requirement "$\Sigma_{j=1}^\infty p_j =$ any finite number" in example 1.)
    \item (Lebesgue measure on all finite semi-open cells in $R^n$)

      Let $S = \{$all finite semi-open cells in $\R^n \}$. We know that $S$ is a semi-ring.

      $\mu(C) := l(I_1) \times l(I_2) \times \ldots \times l(I_n)$, where $l(I) := $length of $I$.

      Then $\mu$ is a $\sigma$-additive measure on $S$.
      \begin{proof}
        We already know that $\mu$ is a measure on the semi-ring $S$. $\mu$ is finitely additive.

        Suppose $A \in S, \{A_j\}_{j=1}^{\infty} \in S, A = \sqcup_{j=1}^{\infty} A_j$.

        WTS: $\mu(A) = \Sigma_{j=1}^{\infty} A_j$

        \underline{Step 1}: $\forall n \in \N, A \supset \sqcup_{j=1}^n A_j$

        $\Rightarrow \Sigma_{j=1}^n \mu(A_j) = \{finit-additivity\} =\mu(\sqcup_{j=1}^n A_j) \leq \mu(A)$

        $\Rightarrow$ Take limit $n \rightarrow \infty$, we have $\Sigma_{j=1}^{\infty} \mu(A_j) \leq \mu(A)$. $\checkmark$

        \underline{Step 2}: Let $A = [\alpha_1,\beta_1) \times \cdots \times [\alpha_n,\beta_n)$ be a finite semi-open cell in $\mathbb{R}^n$, and suppose $A = \bigsqcup_{j=1}^{\infty} A_j,$ where each $A_j$ is also a semi-open cell, and the $A_j$'s are pairwise disjoint.
        
        \underline{Step 2.1}: Partition of $A$ into uniform subcells.
        
        For each integer $m \ge 1$, divide each coordinate interval $[\alpha_i,\beta_i)$ into $m$ equal subintervals: $I^{(m)}_{i,k_i} = \big[\alpha_i + k_i(\beta_i-\alpha_i)/m,\; \alpha_i + (k_i+1)(\beta_i-\alpha_i)/m\big), \qquad k_i = 0,1,\dots,m-1$.
        
        Define the finite family of subcells $\mathcal{Q}_m = \Big\{ Q_{k}^{(m)} = I^{(m)}_{1,k_1} \times \cdots \times I^{(m)}_{n,k_n} : 0 \le k_i \le m-1 \Big\}$.
        
        Then the cells in $\mathcal{Q}_m$ are pairwise disjoint and satisfy $A = \bigsqcup_{Q \in \mathcal{Q}_m} Q$.

        In fact, $|\mathcal{Q}_m| = m^n$, which is finite. By finite additivity of $\mu$, $\mu(A) = \sum_{Q \in \mathcal{Q}_m} \mu(Q)$.
        
        \underline{Step 2.2}: Classification of subcells.
        
        For each $Q \in \mathcal{Q}_m$, there are two possibilities:
        
        1. $Q \subset A_j$ for some $j$;
        
        2. $Q$ intersects at least two distinct sets $A_{j_1}, A_{j_2}$.
        
        Let $\mathcal{Q}_m^{(1)} = \{ Q \in \mathcal{Q}_m : \exists j,\; Q \subset A_j \}, \mathcal{Q}_m^{(2)} = \mathcal{Q}_m \setminus \mathcal{Q}_m^{(1)}$.
        
        Define $A_m^{(1)} = \bigcup_{Q \in \mathcal{Q}_m^{(1)}} Q, A_m^{(2)} = \bigcup_{Q \in \mathcal{Q}_m^{(2)}} Q$.
        
        Then $A = A_m^{(1)} \bigsqcup A_m^{(2)}$, and by finite additivity, $\mu(A) = \mu(A_m^{(1)}) + \mu(A_m^{(2)})$.
        
        \underline{Step 2.3}: Estimate of $\mu(A_m^{(1)})$.
        
        Since every $Q \in \mathcal{Q}_m^{(1)}$ is contained in some $A_j$, and all $Q$’s are disjoint,
        $\mu(A_m^{(1)}) = \sum_{Q \in \mathcal{Q}_m^{(1)}} \mu(Q) \le \sum_{j=1}^{\infty} \mu(A_j)$.
        
        \underline{Step 2.4}: Estimate of $\mu(A_m^{(2)})$.
        
        Each $Q \in \mathcal{Q}_m^{(2)}$ intersects at least two distinct cells $A_{j_1}, A_{j_2}$. Thus, every such $Q$ intersects the boundary of some $A_j$.
        
        Denote $\Gamma = \bigcup_{j=1}^{\infty} \partial A_j$. Each $\partial A_j$ is contained in a finite union of $(n-1)$–dimensional hyperrectangles parallel to the coordinate axes; hence $\Gamma$ is a countable union of such hyperrectangles. Therefore, $\mu(\Gamma) = 0$.
        
        Let $\delta_m = \max_i \frac{\beta_i-\alpha_i}{m}$ be the mesh size of the partition $\mathcal{Q}_m$. Then $A_m^{(2)}$ is contained in the $\delta_m$–neighborhood of $\Gamma$ inside $A$. Because $\Gamma$ has measure zero, for any $\varepsilon > 0$ there exists $\eta > 0$ such that the $\eta$–neighborhood of $\Gamma$ has $\mu$–measure less than $\varepsilon$. For all sufficiently large $m$ (namely $m > (\max_i (\beta_i-\alpha_i))/\eta$), we have $\delta_m < \eta$ and hence $\mu(A_m^{(2)}) < \varepsilon$. This shows $\lim_{m \to \infty} \mu(A_m^{(2)}) = 0$.
        
        Combining above, $\mu(A) = \mu(A_m^{(1)}) + \mu(A_m^{(2)}) le \sum_{j=1}^{\infty} \mu(A_j) + \mu(A_m^{(2)})$, and letting $m \to \infty$ gives $\mu(A) \le \sum_{j=1}^{\infty} \mu(A_j)$. $\checkmark$
      \end{proof}
    \item (Finite Additivive BUT NOT $\sigma$-Additive) 
    
    Let $\Omega = (0,1) \cap \mathbb{Q}.$
    Define the collection $\mathcal{R} = \{A \subset \Omega : A \text{ is finite or co-finite in } \Omega\},$ where “co-finite” means that $\Omega \setminus A$ is finite. Then $\mathcal{R}$ is a ring, since the family of all finite or co-finite subsets of any countable set is closed under finite unions and differences.
    
    Define $\mu : \mathcal{R} \to [0,\infty)$ by $\mu(A) = 0$, if $A$ is finite; $1$, if $A$ is co-finite in $\Omega$.
    
    We verify that $\mu$ is finitely additive.
    
    If $A,B\in\mathcal{R}$ are disjoint, then:

	1.	If both $A$ and $B$ are finite, $A\cup B$ is finite, so $\mu(A\cup B)=0=\mu(A)+\mu(B).$

	2.	If one is finite and the other co-finite, their union is co-finite, so $\mu(A\cup B)=1=\mu(A)+\mu(B).$

	3.	It is impossible for two disjoint co-finite subsets to exist in $\Omega$, so no contradiction arises.
    
    Hence $\mu$ is finitely additive.
    
    Now enumerate $\Omega = \{q_1,q_2,q_3,\dots\}$ and set $A_j = \{q_j\}$.
    
    Then each $A_j$ is finite, hence $\mu(A_j)=0$. Also note that $\Omega = \bigsqcup_{j=1}^{\infty} A_j$.
    
    If $\mu$ were $\sigma$-additive, we would have $\mu(\Omega) = \sum_{j=1}^{\infty} \mu(A_j) = 0$.
    But by definition $\mu(\Omega)=1$. Therefore $\mu$ FAILS to be $\sigma$-additive, even though it is finitely additive.
  \end{enumerate}
\end{example}

\begin{remark}
  A measure $\mu$ with $\sigma$-additivity on $S$ could extend to a measure with $\sigma$-additivity on $\mathcal{R}(S)$ by defining $\mu\left( \bigsqcup_{j=1}^m A_j \right) := \Sigma_{j=1}^m \mu(A_j)$, with $A_j \in S$: disjoint.

  While $\sigma$-additivity on $\mathcal{R}(S)$ can be derived from $\sigma$-additivity on $S$, note that we also have a weaker condition satisfied: \textbf{semi-$\sigma$-additivity}, i.e. $\forall A \subset \cup_{j=1}^\infty A_j, A, A_j \in \mathcal{R}(S), \mu(A) \leq \Sigma_{j=1}^\infty \mu(A_j)$.
\end{remark}

\begin{definition}
    (outer Lebesgue measure of a set E)
    
    Let $\mu$ be a $\sigma$-additive measure on a semi-ring $S$ with unity $\Omega$ (so, $S \subset 2^\Omega$). Let $\mathcal{R}(S) = \mathcal{A}(S)$  --- the minimum algebra containing $S$. For any $E \subset \Omega$, define
    
    \[\mu^*(E) := \inf \left\{ \Sigma_{j=1}^\infty \mu(A_j) : A_j \in S, A \subset \cup_{j=1}^\infty A_j \right\}.\]
    
    Then, $\mu^*$ is called the \textbf{outer(exterior) Lebesgue measure of a set E} induced by $\mu$ on $\Omega$.
\end{definition}

\begin{remark}
  The outer measure $\mu^*$ of a set $E$ always exists (may be infinite), since

  1. $\left\{ \Sigma_{j=1}^\infty \mu(A_j) : A_j \in S, A \subset \cup_{j=1}^\infty A_j \right\}$ at least contains $\Omega$;

  2. Consider the real numbers in $\left\{ \Sigma_{j=1}^\infty \mu(A_j) : A_j \in S, A \subset \cup_{j=1}^\infty A_j \right\}$, they have lower bound $0$. By the completeness of $\R$, the infimum exists.

  \underline{Warning}: 
  
  In general, one CANNOT claim that $\mathcal{A}(S) \supset \mathcal{A}(\Omega)$. This is also the key problem of out outer measure being not able to capture all the information in the algebra generated by $\Omega$!
\end{remark}

\begin{example}
    (An invisible set under the outer measure)
    
    Let $S = \{ [a,b) : a,b \in \mathbb{Q},\, a < b \}$, and define the premeasure $\mu([a,b)) = b - a$. The outer measure $\mu^*$ on $2^{\mathbb{R}}$ is defined by $\mu^*(E) = \inf\{ \sum_{j=1}^{\infty} \mu(A_j) : A_j \in S,\, E \subseteq \bigcup_{j=1}^{\infty} A_j \}$.
    
    Consider the set $E = \mathbb{Q} \cap [0,1]$. We will show that $\mu^*(E) = 1$, while $\mu^*(\{q\}) = 0$ for all $q \in E$. Hence, $\mu^*(\bigsqcup_{q\in E}\{q\}) = 1 > 0 = \sum_{q\in E}\mu^*(\{q\})$, which demonstrates that $\mu^*$ is not countably additive, even for disjoint sets.
\end{example}

\begin{remark}
This example shows that $\mu^*$ cannot "see" the internal structure of sets outside the algebra $\mathcal{A}(S)$. Although $E$ is a countable, measure-zero set in the intuitive sense, any cover of $E$ by rational half-open intervals must in fact cover the entire interval $[0,1]$. Hence, the outer measure treats $E$ as if it were as large as $[0,1]$.
\end{remark}

\begin{remark}
    (The philosophy behind outer measure)
    
    Why do we call it an "outer measure"? 
    
    The name comes from its construction principle: we measure a set \emph{from the outside}. Given a subset $E \subseteq \Omega$, we generally cannot measure $E$ directly, because $E$ may be too irregular or may not belong to the algebra $\mathcal{A}(S)$ where the original measure $\mu$ is defined. Instead, we approximate $E$ by sets $A_j \in S$ that cover $E$ from the outside and take the smallest possible total measure among all such coverings. 
    
    Formally, $\mu^*(E) = \inf\{\sum_j \mu(A_j) : E \subseteq \bigcup_j A_j,\, A_j \in S\}$, which expresses the idea of an \emph{outer approximation}. The measure does not come from the intrinsic structure of $E$, but from the minimal "outer shell" built using measurable sets in $S$. 
    
    Philosophically, $\mu^*$ represents the best information we can obtain about the size of $E$ given our limited "vocabulary" $S$. It is an act of estimation under partial visibility: we look at $E$ through a coarse geometric lens and ask, "How small can the total measure of the covering be if I only use shapes I can measure?"
    
    Thus, it is called an \emph{outer measure} because it always measures from the \emph{outside}, enclosing $E$ within measurable sets rather than dissecting it from the inside.
\end{remark}

\begin{proposition}
  \begin{enumerate}
    \item $\mu^*$ always $\exists$, and $\mu^*(A) \geq 0, \forall A \subset \Omega$.
    \item We can equivalently say in the definition of $\mu^*$ that $A_j$ are disjoint.
    \item $\forall A \in \mathcal{A}(S)$, $\mu(A) = \mu^*(A)$
      \begin{proof}
        On one hand, by the semi-$\sigma$-additivity, $\mu(A) \leq \Sigma_{j=1}^{\infty} \mu(A_j)$ if $\cup_{j=1}^{\infty} A_j \supset A$.

        $\Rightarrow$ Take $inf$: $\mu(A) \leq \mu^*(A)$;

        On the other hand, take the trivial covering: $A_1 = A$,
        
        $\mu(A) = \mu(A_1) = \mu(A_1 \bigsqcup_{j=1}^{\infty}\emptyset) \geq \mu^*(A)$,

        $\Rightarrow \mu(A) = \mu^*(A)$.
      \end{proof}
    \item If $E_1 \subset E_2 \subset \Omega$, then $\mu^*(E_1) \leq \mu^*(E_2)$ (since any covering of $E_2$ is also a covering of $E_1$).
    \item (\emph{Semi-$\sigma$-additivity of $\mu^*$}) 
    
    If $E \subset \cup_{j=1}^\infty E_j$, $E,E_j \subset \Omega$, then $\mu^*(E) \leq \Sigma_{j=1}^\infty \mu^*(E_j)$. (this CANNOT be improved even if $E = \sqcup_{j=1}^{\infty} E_j$ --- check our warning above!)
      \begin{proof}
        $\forall \eps > 0$, 
        
        $\forall j$, choose $\{A_{j,k}\}_{k=1}^{\infty} \subset S$ such that $E_j \subset \cup_{k=1}^{\infty} A_{j,k}$ and

        \(\Sigma_{k=1}^{\infty} \mu(A_{j,k}) \leq \mu^*(E_j) + \frac{\eps}{2^j}\) (thanks to the infimum property).

        Thus, \(E \subset \cup_{j=1}^{\infty} E_j \subset \cup_{j=1}^{\infty} \cup_{k=1}^{\infty} A_{j,k}.\)

        Thus, by the definition of $\mu^*$ and semi-$\sigma$-additivity of $\mu$,

        \(\mu^*(E) \leq \Sigma_{j=1}^{\infty} \Sigma_{k=1}^{\infty} \mu(A_{j,k}) \leq \Sigma_{j=1}^{\infty} \left( \mu^*(E_j) + \frac{\eps}{2^j} \right) = \Sigma_{j=1}^{\infty} \mu^*(E_j) + \eps.\)

        Let $\eps \rightarrow 0^+$, we get the desired result.
      \end{proof}
  \end{enumerate}
\end{proposition}

\begin{example}
  Let's fix a bounded cell $\Omega$ in $\R^d$. Let $S = \{$all cells $C \subset \Omega\}$.

  Define $\mu(\{p\}) = 0$ for all $p \in \Omega$. Consider $E = \Omega \cap \Q^n, E = \{q_1, q_2, \ldots\}$

  $\Rightarrow \mu^*(E) \leq \Sigma_{j=1}^{\infty} \mu^*(\{q_j\}) = \Sigma_{j=1}^{\infty} \mu(\{q_j\}) = 0 \Rightarrow \mu^*(E) = 0$.

  $\mu^*(\Omega \setminus E) \leq \mu^*(\Omega) = \mu(\Omega)$

  But by semi-$\sigma$-additivity,
  \(\mu(\Omega) = \mu^*(\Omega) \leq \mu^*(E) + \mu^*(\Omega \setminus E) = \mu^*(\Omega \setminus E).\)

  $\Rightarrow \mu^*(\Omega \setminus E) = \mu(\Omega)$, which means that the outer measure CANNOT distinguish the counterble but sparce set $\Q^n$.
\end{example}

\begin{definition}
  Let $S$ be a semi-ring with unity $\Omega$, and $\mu$ be a $\sigma$-additive measure on $S$. $R(S) = \mathcal{A}(S)$ --- the minimum algebra containing $S$, $\mathcal{A}(S) \subset 2^\Omega$. A set $E \subset \Omega$ is called \textbf{(Lebesgue) measurable} if $\forall \eps > 0$, $\exists B_{\eps} \in \mathcal{A}(S)$ such that $\mu^*(E \triangle B_{\eps}) < \eps.$
\end{definition}

\begin{example}
  In this setting, let $\mu^*(E) = 0$, then $E$ is measurable: Choose $B_{\eps} = \emptyset$, then \(\mu^*(E \triangle B_{\eps}) = \mu^*(E) < \eps\).
\end{example}

\begin{remark}
    The definition of a (Lebesgue) measurable set captures the idea of \emph{approximability by “nice” sets}. A set $E \subset \Omega$ is called measurable if it can be arbitrarily well approximated by sets $B_\varepsilon$ from the algebra $\mathcal{A}(S)$, in the sense that the “disagreement region” between $E$ and $B_\varepsilon$, namely the symmetric difference $E \triangle B_\varepsilon$, has arbitrarily small outer measure: $\mu^*(E \triangle B_\varepsilon) < \varepsilon$ for all $\varepsilon > 0$.
    
    Intuitively, this means that even if $E$ itself may be irregular or complicated, we can always find a clean, measurable set $B_\varepsilon$ that almost coincides with $E$ up to an arbitrarily small “error area.” Measurable sets are precisely those whose geometry can be faithfully captured through such approximations.
    
    In the above example, if $\mu^*(E)=0$, then $E$ is trivially measurable. Indeed, we can take $B_\varepsilon = \emptyset$, so that $\mu^*(E \triangle B_\varepsilon) = \mu^*(E) = 0 < \varepsilon$. This illustrates that every \emph{measure-zero set} is measurable: such sets are geometrically “invisible” to the outer measure, since they can be ignored without affecting any measured quantity.
\end{remark}

\underline{Setting}: 

$(\Omega, S, \mu)$ --- $\Omega$ - set, $S$ - semi-ring with unity $\Omega$, $\mu$ - $\sigma$-additive measure on $S$

$\rightarrow$ directly extend to $(\Omega, \mathcal{A}(S), \mu)$

$\rightarrow$ introduce $\mu^*$ on the whole $2^\Omega$

$\rightarrow$ $(\Omega, \mathcal{M}(\Omega), \mu)$, with $\mathcal{M}(\Omega)$: collection of all measurable sets in $\Omega$.

'measurable': $\forall A \in \mathcal{M}(\Omega)$, $\forall \eps > 0$, $\exists B_{\eps} \in \mathcal{A}(S)$ such that $\mu^*(A \triangle B_{\eps}) < \eps.$

\begin{remark}
  To better distinguish $\mu$ and $\mu^*$, for those in the original $\mathcal{A}(S)$, we use $\mu$. Otherwise, we use the notation $\mu^*$. Thus, $*$ emphases that the measure on the set is defined by extanding $\mu$.
\end{remark}

\begin{theorem}
  In the above setting, let $\mathcal{M}(S)$ be the collection of all measurable sets and we set $\mu(A) := \mu^*(A), \forall A \in M(S)$. Then,
  \begin{enumerate}
    \item $\mathcal{M}(S)$ is a $\sigma$-algebra.

      ($M(S)$ extends the original algebra $\mathcal{A}(S)$.)
    \item $\mu^*$ is $\sigma$-additive on $\mathcal{M}(S)$.

      ($M$ extends the original measure $\mu$ on $\mathcal{A}(S)$.)
  \end{enumerate}
  \begin{proof}

    First of all, we know that $\Omega \in \mathcal{M}(\Omega)$.


    \underline{Step I}: prove if $A \in \mathcal{M}(\Omega)$, then $\Omega \setminus A \in \mathcal{M}(\Omega)$.

    Fix $\eps > 0$, $\exists B_{\eps} \in \mathcal{A}(S)$ such that $\mu^*(A \triangle B_{\eps}) < \eps.$ 
    
    Consider $\Omega \setminus B_{\eps} \in \mathcal{A}$. Then, note $(\Omega \setminus A) \triangle (\Omega \setminus B_{\eps}) = A \triangle B_{\eps}$.

    Thus, $\mu^*((\Omega \setminus A) \triangle (\Omega \setminus B_{\eps})) < \eps$ $\Rightarrow \Omega \setminus A \in \mathcal{M}(\Omega)$.

    \underline{Step II}: prove $\forall A_1, A_2, \ldots, A_n \in \mathcal{M}(\Omega)$, we have $\bigcup_{i=1}^n A_i \in \mathcal{M}(\Omega)$.

    Only need to prove for $n=2$ (others by induction).

    $A_1, A_2 \in \mathcal{M}(\Omega)$, $\forall \eps > 0. \exists B_1, B_2 \in \mathcal{A}: \mu^*(A_1 \triangle B_1) < \eps, \mu^*(A_2 \triangle B_2) < \eps$.

    $A = A_1 \bigcup A_2$, we will approxmate by $B = B_1 \bigcup B_2$.
    
    Since $(A_1 \bigcup A_2) \triangle (B_1 \bigcup B_2) \subset (A_1 \bigcup B_1) \triangle (A_2 \bigcup B_2)$,

    $\mu^*(A \triangle B) < \mu^*(A_1 \triangle B_1) + \mu^*(A_2 \triangle B_2) < 2\eps$

    $\Rightarrow A_1 \bigcup A_2 \in \mathcal{M}(\Omega)$.

    Thus, the first statement is proved.

    \begin{corollary}
    $\mathcal{M}(\Omega)$ is an algebra.
        
        \begin{proof}
            \begin{itemize}
                \item contains $\Omega$.
                \item closed under taking union: proved above.
                \item closed under intersection:
                \item closed under symmetric difference: $A \triangle B = $
            \end{itemize}
        \end{proof}
    \end{corollary}

    \underline{Step III}: prove $\mu^*$ is finitely additive on $\mathcal{M}(\Omega)$.

    So, $\forall A_1, A_2, \ldots, A_n \in \mathcal{M}(\Omega)$, we need to show $\mu(A_1 \cup A_2 \cup \cdots \cup A_n) = \mu(A_1) + \mu(A_2) + \cdots + \mu(A_n)$.

    Similarly, only need to show for $n=2$.

    Take $A_1, A_2 \in \mathcal{M}(\Omega), A_1 \cap A_2 = \emptyset$.

    $\forall \eps > 0, \exists B_1, B_2 \in \mathcal{A}(S): \mu^*(A_1 \triangle B_1) < \eps, \mu^*(A_2 \triangle B_2) < \eps$.

    Since $B_1 \bigcap B_2 \subset (A_1 \triangle B_1) \bigcup (A_2 \triangle B_2)$, we have $\mu^*$




    \underline{Step IV}: prove $\mu^*$ is a $\sigma$-algebra on $\mathcal{M}(\Omega)$.



    \dots

    \dots


    

    Replace by disjoint union: let $A_1^{\prime} = A_1, A_2^{\prime} = A_2 \setminus A_1, A_3^{\prime} = A_3 \setminus (A_1 \bigcup A_2), \ldots$.

    Then, we have $A = \bigsqcup_{i=1}^{\infty}A_i^{\prime}$.

    We have 







    \underline{Step V}: prove $\mu^*$ is $\sigma$-additive on $\mathcal{M}(\Omega)$.

    WTS: $\forall A_1, A_2, \ldots, A_n \in \mathcal{M}(\Omega)$, we have $\mu^*(A_1 \cup A_2 \cup \cdots \cup A_n) = \mu^*(A_1) + \mu^*(A_2) + \cdots + \mu^*(A_n)$.
    
    


  \end{proof}
\end{theorem}




Conclusion: We end up with a triple ($\Omega, \mathcal{M}(\Omega), \mu$) --- (set: $\Omega$, $\sigma$-algebra: $\mathcal{M}(\Omega)$, $\sigma$-additive measure on $\mathcal{M}(\Omega)$: $\mu$).

\begin{definition}
    (Measure Space)

    Such a triple ($\Omega, \mathcal{A}, \mu$) ($\mathcal{A}$ is some $\sigma$-algebra on the set $\Omega$) is called a \textbf{measure space (spcae with measure)}.
\end{definition}



\clearpage


\subsection{Measure Space}
Recall: A triple ($\Omega, \mathcal{A}, \mu$) ($\mathcal{A}$ is some $\sigma$-algebra on the set $\Omega$) is called a \textbf{measure space (spcae with measure)}.

\begin{definition}
    (Completeness of a Measure Space)


    A \textbf{complete} measure (or, more precisely, a complete measure space) is a measure space in which every subset of every null set is measurable (having measure zero).

    More formally, if ($\Omega, \mathcal{A}, \mu$) is a measure space, then it's called \textbf{complete} if and only if $A \subset E \in \mathcal{A}$, $\mu(E) = 0$, $\Rightarrow A \in \mathcal{A}$ (and hence $\mu(A)=0$).
\end{definition}

\begin{example}
    For ($\Omega, \mathcal{M}(\Omega), \mu$), we always have completeness:

    $\mu(A) = 0, E \subset A \Rightarrow 0 \leq \mu^*(E) \leq \mu^*(A) = 0 \Rightarrow E \in \mathcal{M}(\Omega)$.

    But this FAILS in general. For example, $\exists$ measure 0 non-Borel sets, which is contained in some measure 0 Borel sets, so Lebesgure measure $\mu$ on $\R^n$, restricted to Borel $\sigma$-algebra is incomplete.

    However, any incomplete measure space can extend its measure to attain a complete measure space. One just need to follow the Lebesgue extension of a general measure space ($\Omega, \mathcal{A}, \mu$).
\end{example}

\begin{theorem}
    For any measure space ($\Omega, \mathcal{A}, \mu$), the following holds:
    \begin{enumerate}
        \item $\forall A_1 \subset A_2 \subset A_3 \subset \ldots$ with $A_i \in \mathcal{A}$, $\mu(\bigcup_{i=1}^{\infty} A_i) = \lim\limits_{i \to \infty} \mu(A_i)$.
        \item If $A_1 \supset A_2 \supset A_3 \supset \ldots$, $A_i \in \mathcal{A}$, then $\lim\limits_{i \to \infty} \mu(A_i) = \mu(\bigcap_{i=1}^{\infty} A_i)$.
    \end{enumerate}

    Both $1$ and $2$ are called the \textbf{continuity} of the measure.

    \begin{proof}
    
    \end{proof}
\end{theorem}

\underline{Question}: What about $\infty$-valued measures?

Consider a space with measure ($\Omega, \mathcal{A}, \mu$), where $\mu$ is a $\bar{\R}$-valued measure. The definition of finite additivity and $\sigma$-additivity is repeated word-by-word:

\textbf{Finite additivity}:
\begin{enumerate}
  \item $\mu(A) \geq 0$;
  \item $\mu(\bigcup_{j=1}^{n} A_j) = \Sigma_{j=1}^{n} \mu(A_j)$;
\end{enumerate}

\textbf{$\sigma$-additivity}:
\begin{enumerate}
  \item $\mu(A) \geq 0$;
  \item $\mu(\bigcup_{j=1}^{\infty} A_j) = \Sigma_{j=1}^{\infty} \mu(A_j)$;
\end{enumerate}

Then we easily deduce several similar properties.

\begin{proposition}
    \begin{enumerate}
        \item $\mu(\emptyset) = 0$.
        \item If $A \subset B,$ then $\mu(A) \leq \mu(B)$.
        \item If $A \subset \bigcup_{j=1}^{\infty} A_j,$ then $\mu(\bigcup_{j=1}^n A_j) \leq \Sigma_{j=1}^{\infty} \mu(A_j)$.
    \end{enumerate}
\end{proposition}

\begin{definition}
    A measure space with $\infty$-valued measure is called \textbf{$\sigma$-finite} if 
    
    $\Omega = \bigsqcup_{k=1}^{\infty} \Omega_k, \Omega_k \in \mathcal{A}$, $\mu(\Omega_k) < \infty$. 
    
    Then $\forall A \subset \mathcal{A}$, $\mu(A) = \Sigma_{k=1}^{\infty} \mu(A \bigcap \Omega_k) =: \Sigma_{k=1}^{\infty} \mu_k(A)$.
\end{definition}

So, essentially, $\mu$ is obtained from $\{\mu_k\}_{k=1}^{\infty}$, with each $\mu_k$ defined on $\mathcal{A}\cap 2^{\Omega_k}$.









\begin{example}
    $\R^n = \bigsqcup_{i_1, i_2, \ldots, i_n}[i_1, i_1+1) \times [i_2, i_2+1) \times \ldots \times [i_n, i_n+1), i_1, \ldots, i_n \in \Z$.

\end{example}










\begin{remark}
  For $\Omega = \bigsqcup_{k=1}^{\infty} \Omega_k = \bigsqcup_{j=1}^{\infty} \Omega_j^{\prime}$

  Mesures agree:





\end{remark}










\clearpage






\subsection{Lebesgue Measure in $\R^n$}

\underline{Goal}: Understand $\mathcal{M}(\R^n)$.

\underline{Main fact}: $\mathcal{B}(\R^n) \subsetneq \mathcal{M}(\R^n) \subsetneq 2^{\R^n}$.







\begin{proposition}
  \begin{enumerate}
    \item (Shift-invariance) If $E_{\alpha} := \{x+\alpha, x\in E, \alpha \in \R^n$: fixed$\}$, then 
    
    $E_{\alpha} \in \R^n \Leftrightarrow E \subset \mathcal{M}(R)$; and we also have $\mu(E_{\alpha} = \mu(E))$.

    It holds since it holds for cells.

    \item $\mathcal{B}(\R^n) \subset \mathcal{M}(\R^n)$
  \end{enumerate}

\end{proposition}

\begin{proposition}
  $\exists$ a non-measurable subset $A \subset [0,1)$.
  \begin{proof}
    On $[0,1)$, consider the following equivalent relation:


  \end{proof}
\end{proposition}





\begin{proposition}
  $\forall A \subset \R$ with $\mu(A)>0$, $A$ contains some $B \subset A$ s.t. $B \notin \mathcal{M}(\R)$.
  \begin{proof}
  
  \end{proof}

\end{proposition}

\begin{remark}
  The same holds in $\R^n$: $\forall A \subset \R^n$ with $\mu(A)>0$, $A$ contains some $B \subset A$ s.t. $B \notin \mathcal{M}(\R^n)$.
\end{remark}






\clearpage



\subsection{Cantor Set}

We build a sequence of sets:

$E_0 = [0,1]$

$E_1 = E_0 \setminus I_1$, $I_1 = (\frac{1}{3}, \frac{2}{3})$.

$E_2 = E_1 \setminus I_2$, $I_2 = I_{1,1} \bigcup I_{1,2}$, $I_{1,1} = (\frac{1}{9}, \frac{2}{9})$, $I_{1,2} = (\frac{7}{9}, \frac{8}{9})$.

$E_3 = E_2 \setminus I_3$, $I_3 = I_{2,1} \bigcup I_{2,2} \bigcup I_{2,3} \bigcup I_{2,4}$, $I_{2,1} = (\frac{1}{27}, \frac{2}{27})$, $I_{2,2} = (\frac{4}{27}, \frac{5}{27})$, $I_{2,3} = (\frac{19}{27}, \frac{20}{27})$, $I_{2,4} = (\frac{25}{27}, \frac{26}{27})$.


$\ldots$

We get a sequence of sets $\{E_k\}$, $\forall E_k$ is closed. $\Rightarrow C_0 := \bigcap_{k=1}^{\infty}E_k$, $C_0$ is closed and bounded $\Rightarrow$ $C_0$ is compact.

\begin{definition}
  (Cantor Set)

  Such $C_0$ is called a \textbf{(standard) Cantor set}.
\end{definition}

\begin{proposition}
  \begin{enumerate}
    \item $C_0$ is compact and $C_0 \subset [0,1]$;
    \item $C_0$ is nowhere dense;
    \item $\mu(C_0) = 0$
    \begin{proof}
    
    \end{proof}
    \item $C_0$ is continual.
    \begin{proof}
    
    \end{proof}
  \end{enumerate}
\end{proposition}



\clearpage


\subsection{Cantor Staircase Function}

\begin{definition}
  
\end{definition}


\begin{lemma}
  Let $f: \Omega \rightarrow \Omega^{\prime}, S^{\prime} \subset 2^{\Omega^{\prime}}$, then $\mathcal{A}(f^{-1}*(S^{\prime})) = f^{-1}(\mathcal{A}(S^{\prime}))$.
  \begin{proof}
  
  \end{proof}
\end{lemma}

\begin{corollary}
  (Preimage of Borel set is Borel.)

  If $f: [a,b] \rightarrow [c,d]$ is continuous, then $f^{-1}(E^{\prime})$ is Borel, provided $E^{\prime} \subset [c,d]$ is Borel.
  \begin{proof}
    Follows from $f^{-1}(G)$ is open if $G$ is open.
  \end{proof}
\end{corollary}


Now, consider $\phi (x) := x + K(x)$, $\phi : [0,1] \rightarrow [0,2]$, $\phi$ is strictly increasing.


\end{document}
